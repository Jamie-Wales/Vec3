Alongside standard arithmetic operations, variable assignment, function definitions, and plotting the language also 
supports the following interesting features:

\paragraph{User defined operators} The user can define their own operators, both unary and binary, with the 
following syntax:

\begin{minted}{fsharp}
// Define a binary operator
let (|+|) = (a: float, b: float) -> a + b

// Define a unary operator
let (|!|) = (a: bool) -> not(a)
\end{minted}

This is a powerful feature allowing for the language to be extended in a way that is meaningful to the user.

\paragraph{Static type inference} The language uses Hindley-Milner type inference to infer the types of
expressions while giving the user the option of specifying types explicitly, improving the correctness of the
code and reducing the cognitive load on the user.

\begin{minted}{fsharp}
let f = (x) -> x + 1.0
let f = (x: float) -> x + 1.0 // Explicit type annotation
let f: (float) -> float = (x) -> x + 1.0 // Alternative explicit type annotation
let f = (x: float) : float -> x + 1.0 // Alternative explicit type annotation
\end{minted}

\paragraph{Compound data types} The language supports compound data types such as lists, tuples and records,
allowing for more complex data structures to be represented.

\begin{minted}{fsharp}
let l = [1, 2, 3]
let l1 = l[0]

let t = (1, true, "this is a tuple")

let r = { x = 1, y = 2 }
let r1 = r.x
\end{minted}

\paragraph{Imports} The language supports importing other files, allowing for code to be split across multiple
files, improving code organisation and reusability.
As well as this, a small standard library has been written that can be imported as desired.

\begin{minted}{fsharp}
import "file.vec3"
import "list"

let f = (x) -> func(x) // func defined in file.vec3
let l = [1, 2, 3]
let l1 = head(l) // head defined in list
\end{minted}

\paragraph{Turing Completeness} The language is Turing complete, allowing for the implementation of any
algorithm that can be implemented in a Turing complete language.
As the language is immutable, this is done through recursion and higher order functions, with control flow
managed through if expressions.

\begin{minted}{fsharp}
let rec map = (list, f) -> if len(list) == 0 then [] else f(head(list)) :: map(tail(list), f)
\end{minted}

\paragraph{Calculus} The language has built-in functions for finding the integral function, derivative function
and tangent function at a given point of a function.
As such, a few useful math operations are built around these, such as finding the integral:

\begin{minted}{fsharp}
let findIntegral = (f, a, b) {
    let integral = integrate(f)
    integral(b) - integral(a)
}
\end{minted}

\paragraph{Vector and Matrix operations} The language represents vectors and matrices as lists of numeral types, with
operations such as addition, subtraction, multiplication and division defined for them, and functions such as \textit{transpose} and \textit{determinant} defined for matrices.

\begin{minted}{fsharp}
let v1 = [1.0, 2.0, 3.0]
let v2 = [4.0, 5.0, 6.0]
let v3 = v1 + v2
let dot = v1 .* v2
let cross = v1 X v2

let m1 = [[1.0, 2.0], [3.0, 4.0]]
let trans = transpose(m1)
let det = determinate(m1)
let inv = inverse(m1)
\end{minted}

\paragraph{Casting} The language has first class support for casting between types, allowing for the user to
explicitly cast between types as desired.
A useful example of this is casting integer lists to float lists:

\begin{minted}{fsharp}
let l = [1..10] : [float]
\end{minted}

\paragraph{Floats} The language has built-in support for floating point numbers, allowing for the user to perform
operations on them as desired.
They can be created either with a decimal point or with scientific notation as follows:

\begin{minted}{fsharp}
let f1 = 1.0
let f2 = 1e-3
\end{minted}

\paragraph{Complex numbers} The language has built-in support for complex numbers, allowing for the user to
perform operations on them as desired.
They are created by adding an imaginary unit to a number, as follows:

\begin{minted}{fsharp}
let c1 = 1 + 2i
let c2 = 3 - 4i
let c3 = c1 + c2
\end{minted}

\paragraph{Rationals} The language has built-in support for rational numbers, allowing for the user to perform
operations on them as desired, reducing the risk of floating point errors.
They are created by dividing two integers with no spaces between them, as follows:

\begin{minted}{fsharp}
let r1 = 1/2
let r2 = 3/4
let r3 = r1 + r2
\end{minted}



\paragraph{Async functions} The language supports asynchronous functions, allowing for non-blocking operations
to be performed.
This is useful is a long calculation is to be performed, as the function will run in the background until awaited.

\begin{minted}{fsharp}
let async longCalc = sum([1..1000000])

let result = await(longCalc)
\end{minted}

\paragraph{Range Expressions} The language supports range expressions, allowing for a list of numbers to be
generated easily.
In addition to this, list indexing also supports ranges, allowing for a sublist to be extracted from a list.

\begin{minted}{fsharp}
let l = [1..10]
let l1 = l[1..5] // Extracts the sublist [2, 3, 4, 5, 6]
let l2 = l[1..] // Extracts the sublist [2, 3, 4, 5, 6, 7, 8, 9]
let l3 = l[..5] // Extracts the sublist [1, 2, 3, 4, 5]
\end{minted}

\paragraph{Function Plotting} The language has built-in functions for plotting functions, allowing for easy
visualisation of functions.
This is done through the \textit{plotFunc} function, which takes in a function and plots it on a graph.

\paragraph{Interactive Plotting} The plot windows are interactive, with the user being able to zoom in and out,
move around and adjust the axes as desired, as well as inputting functions to be plotted on command.