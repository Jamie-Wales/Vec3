\section{Sprint 1: Basic expressions}\label{sec:sprint-1:-basic-expressions}

This sprint focused on implementing a lexer and parser for the language, with precedence rules for the arithmetic 
operators, parsed with Pratt parsing.

\subsection{Grammar in BNF}\label{subsec:grammar-in-bnf1}

\begin{verbatim}
<expr> ::= <term> | <term> '+' <expr> | <term> "-" <expr>
<term> ::= <factor> | <factor> "*" <term> | <factor> "/" <term>
<factor> ::= <number> | "(" <expr> ")"
<number> ::= <int> | <float>
<int> ::= <digit> | <digit> <int>
<float> ::= <int> "." <int>
<digit> ::= "0" | "1" | "2" | "3" | "4" | "5" | "6" | "7" | "8" | "9"
\end{verbatim}

\section{Sprint 2: Variable assignment}\label{sec:variable-assignment}

In this sprint we added variable assignment to the parser, with the ability to bind an expression to a variable 
name, as well as a few new operators such as \textit{==} for equality and \textit{\%} for modulo, as well as 
unary operators.

Variable bindings are in the form \texttt{let \textit{identifier} = \textit{expr}}.

\subsection{Grammar in BNF}\label{subsec:grammar-in-bnf2}

\begin{verbatim}
<stmtlist> ::=  <stmt> 
              | <stmt> <stmtlist>
<stmt> ::=  <expr> 
          | "let" <identifier> "=" <expr>

<expr> ::=  <term> 
          | <term> "+" <expr> 
          | <term> "-" <expr>
          | <term> "==" <expr>
          | <term> "!=" <expr>
          | <term> "<" <expr>
          | <term> ">" <expr>
          | <term> "<=" <expr>
          | <term> ">=" <expr>
          | <term> "&&" <expr>
          | <term> "||" <expr>
<term> ::=  <factor> 
          | <factor> "*" <term> 
          | <factor> "/" <term> 
          | <factor> "%" <term> 
<factor> ::=  <number> 
            | <identifier> 
            | <unaryop> <factor>
            | "(" <expr> ")" 
            | <factor> "^" <factor>

<unaryop> ::= "-" | "!" | "+" | <userop>

<number> ::= <int> | <float>
<int> ::= <digit> | <digit> <int>
<float> ::= <int> "." <int>
<digit> ::= "0" | "1" | "2" | "3" | "4" | "5" | "6" | "7" | "8" | "9"

<identifier> ::= <letter> | <letter> <identifier>
<letter> ::=  "a" | "b" | "c" | "d" | "e" | "f" | "g" | "h" | "i" | "j" | "k" | "l" | "m" 
            | "n" | "o" | "p" | "q" | "r" | "s" | "t" | "u" | "v" | "w" | "x" | "y" | "z"
\end{verbatim}

\section{Sprint 3: Interpreter}\label{sec:interpreter}

In this sprint a basic interpreter was implemented, with the ability to evaluate expressions and variable bindings.
We used a simple environment to store variable bindings (a map of \textit{string} name to \textit{expr}), and a 
recursive evaluation function to evaluate expressions.
It was a REPL style interpreter, where the last expression of a statement list was evaluated and printed, and ran through
the command line.
We did not change the grammar in this sprint.

\section{Sprint 4: Functions}\label{sec:functions}

In this sprint we added the ability to define and call functions, with a simple lambda syntax of the form 
\texttt{(\textit{args}) -> \textit{expr}} and function calls of the form \texttt{\textit{funcName / lambda}(\textit{args})}.
Call by value semantics were used, with a new environment created for each function call, consisting of the arguments
bound to the parameter names and the parent environment.

We also added an \textit{assert} statement, allowing for simple tests to be written in the language and support for 
rational numbers was added.

Rational numbers are in the form \texttt{1/2} (note the lack of spaces around the \texttt{/}), and allowed for more 
precision in arithmetic operations, compared to floating point numbers.

\subsection{Grammar in BNF}\label{subsec:grammar-in-bnf4}

\begin{verbatim}
<stmtlist> ::=  <stmt> 
              | <stmt> <stmtlist>
<stmt> ::=  <expr> 
          | "let" <identifier> "=" <expr>

<expr> ::=  <term> 
          | <term> "+" <expr> 
          | <term> "-" <expr>
          | <term> "==" <expr>
          | <term> "!=" <expr>
          | <term> "<" <expr>
          | <term> ">" <expr>
          | <term> "<=" <expr>
          | <term> ">=" <expr>
          | <term> "&&" <expr>
          | <term> "||" <expr>
<term> ::=  <factor> 
          | <factor> "*" <term> 
          | <factor> "/" <term> 
          | <factor> "%" <term> 
<factor> ::=  <number> 
            | <unaryop> <factor>
            | <identifier> 
            | "(" <expr> ")" 
            | <factor> "^" <factor>
            | <factor> "(" <exprlist> ")"
            | <lambda>
<lambda> ::= "(" <exprlist> ")" "->" <expr>

<unaryop> ::= "-" | "!" | "+" | <userop>
            
<number> ::= <int> | <float> | <rational>
<int> ::= <digit> | <digit> <int>
<float> ::= <int> "." <int>
<rational> ::= <int> "/" <int>
<digit> ::= "0" | "1" | "2" | "3" | "4" | "5" | "6" | "7" | "8" | "9"

<identifier> ::= <letter> | <letter> <identifier>
<letter> ::=  "a" | "b" | "c" | "d" | "e" | "f" | "g" | "h" | "i" | "j" | "k" | "l" | "m" 
            | "n" | "o" | "p" | "q" | "r" | "s" | "t" | "u" | "v" | "w" | "x" | "y" | "z"
            
<exprlist> ::= <expr> | <expr> "," <exprlist>

\end{verbatim}

\section{Sprint 5: Static type checking}\label{sec:static-type-checking}

In this sprint we added static type checking to the language, with a simple type inference system based on 
Hindley-Milner.

We go over this system in more detail in section~\ref{sec:type-inference}.

The concept of types was introduced, with the types \texttt{Int}, \texttt{Float}, \texttt{Bool}, \texttt{Function} and
\texttt{Never}.

\subsection{Grammar in BNF}\label{subsec:grammar-in-bnf5}

\begin{verbatim}
<stmtlist> ::=  <stmt> 
              | <stmt> <stmtlist>
<stmt> ::=  <expr> 
          | "let" <identifier> "=" <expr>
          | "let" <identifier> ":" <type> "=" <expr>

<expr> ::=  <term> 
          | <term> "+" <expr> 
          | <term> "-" <expr>
          | <term> "==" <expr>
          | <term> "!=" <expr>
          | <term> "<" <expr>
          | <term> ">" <expr>
          | <term> "<=" <expr>
          | <term> ">=" <expr>
          | <term> "&&" <expr>
          | <term> "||" <expr>
<term> ::=  <factor> 
          | <factor> "*" <term> 
          | <factor> "/" <term> 
          | <factor> "%" <term> 
<factor> ::=  <number> 
            | <identifier> 
            | <unaryop> <factor>
            | "(" <expr> ")" 
            | <factor> "^" <factor>
            | <factor> "(" <exprlist> ")"
            | <bool>
            | <lambda>
<lambda> ::=  "(" <typedexprlist> ")" "->" <expr>
            | "(" <typedexprlist> ")" ":" <type> "->" <expr>


<unaryop> ::= "-" | "!" | "+" | <userop>

<bool> ::= "true" | "false"
            
<number> ::= <int> | <float> | <rational>
<int> ::= <digit> | <digit> <int>
<float> ::= <int> "." <int>
<rational> ::= <int> "/" <int>
<digit> ::= "0" | "1" | "2" | "3" | "4" | "5" | "6" | "7" | "8" | "9"

<identifier> ::= <letter> | <letter> <identifier>
<letter> ::=  "a" | "b" | "c" | "d" | "e" | "f" | "g" | "h" | "i" | "j" | "k" | "l" | "m" 
            | "n" | "o" | "p" | "q" | "r" | "s" | "t" | "u" | "v" | "w" | "x" | "y" | "z"
            
<exprlist> ::= <expr> | <expr> "," <exprlist>
<typedexprlist> ::=  <expr> ":" <type> 
                   | <expr> ":" <type> "," <typedexprlist>
                   | <expr> "," <typedexprlist>
    
<type> ::= "int" | "float" | "bool" | "(" <typelist> ")" "->" <type>

<typelist> ::= <type> | <type> "," <typelist>
\end{verbatim}

\section{Sprint 6: Bytecode}\label{sec:bytecode}

In this sprint the interpreter was rewritten to use a bytecode interpreter, with a stack based virtual machine as 
well as a simple bytecode compiler, allowing for more efficient evaluation of expressions.

The VM and compiler were written in F\# and used a simple stack based architecture, with a small instruction set.
Additionally, in order to simplify the VM, binary and unary operators were implemented as functions in the language, 
instead of separate AST nodes for \textit{BinaryOp} and \textit{UnaryOp}.
This helped to simplify not only the AST and the bytecode, but the architecture of the whole system.
For example, it prevented a lot of code repetition during type inference.

The grammar was not changed in this sprint.

\section{Sprint 7: GUI}\label{sec:gui}

A simple GUI was developed in order to allow easier testing of the language, with a text box for input and output and a 
decompiler output for debugging.
The GUI was written in F\# using Avalonia\citep{avalonia}.
We did not change the grammar in this sprint.

\section{Sprint 8: Plotting}\label{sec:plotting1}

In this sprint we added the ability to plot lists of points, with a simple plotting function that took a list of
\textit{x} coordinates and a list of \textit{y} coordinates and plotted them on a graph using \citet{scottPlot}.
Naturally, we had to add a new type, \texttt{List}, to the language, and as an extension of this, we added the ability to
define lists using the syntax \texttt{[1, 2, 3, 4]}.
Lists are type encoded with their length and the type of their elements, and are immutable.
A few functions were added such as \textit{dot} and \textit{cross} for vector operations, which both make use of the 
length encoding to prevent runtime errors.

Other compound data types such as tuples and records were also added, with the syntax \texttt{(1, 2, 3)} and
\texttt{\{x=1, y=2\}} respectively.

\subsection{Grammar in BNF}\label{subsec:grammar-in-bnf8}

\begin{verbatim}
<stmtlist> ::=  <stmt> 
              | <stmt> <stmtlist>
<stmt> ::=  <expr> 
          | <vardecl>
          | <assertion>
          
<vardecl> ::= "let" <identifier> "=" <expr>
            | "let" <identifier> ":" <type> "=" <expr>
            
<assertion> ::= "assert" <expr> | "assert" <expr> <string>
                

<expr> ::=  <term> 
          | <term> "+" <expr> 
          | <term> "-" <expr>
          | <term> "==" <expr>
          | <term> "!=" <expr>
          | <term> "<" <expr>
          | <term> ">" <expr>
          | <term> "<=" <expr>
          | <term> ">=" <expr>
          | <term> "&&" <expr>
          | <term> "||" <expr>
<term> ::=  <factor> 
          | <factor> "*" <term> 
          | <factor> "/" <term> 
          | <factor> "%" <term> 
<factor> ::=  <literal> 
            | "(" <expr> ")" 
            | <factor> "^" <factor>
            | <unaryop> <factor>
            | <factor> "(" <exprlist> ")"
            | <factor> "." <identifier>
            | <factor> "[" <expr> "]"
            | <factor> "." <identifier>
            | <factor> "[" <expr> ":" <expr> "]"
            | <factor> "[" <expr> ":" "]"
            | <factor> "[" ":" <expr> "]"

<unaryop> ::= "-" | "!" | "+" | <userop>

<literal> ::= <number> | <identifier> | <bool> | <list> | <lambda> | <string> | "()" | <tuple> | <record>
<string> ::= '"' <charlist> '"' | '""'
<charlist> ::= <char> | <char> <charlist>
<list> ::= "[" <exprlist> "]"
<tuple> ::= "(" <exprlist> ")"
<record> ::= "{" <recordlist> "}"
<recordlist> ::=  <identifier> "=" <expr> 
                | <identifier> "=" <expr> "," <recordlist>
                | <identifer> ":" <type> "=" <expr> 
                | <identifier> ":" <type> "=" <expr> "," <recordlist>
            
<lambda> ::=  "(" <typedexprlist> ")" "->" <expr>
            | "(" <typedexprlist> ")" ":" <type> "->" <expr>

<bool> ::= "true" | "false"
            
<number> ::= <int> | <float> | <rational>
<int> ::= <digit> | <digit> <int>
<float> ::= <int> "." <int>
<rational> ::= <int> "/" <int>
<digit> ::= "0" | "1" | "2" | "3" | "4" | "5" | "6" | "7" | "8" | "9"

<identifier> ::= <letter> | <letter> <identifier>
<letter> ::=  "a" | "b" | "c" | "d" | "e" | "f" | "g" | "h" | "i" | "j" | "k" | "l" | "m" 
            | "n" | "o" | "p" | "q" | "r" | "s" | "t" | "u" | "v" | "w" | "x" | "y" | "z"

<exprlist> ::= <expr> | <expr> "," <exprlist>
<typedexprlist> ::=  <expr> ":" <type> 
                   | <expr> ":" <type> "," <typedexprlist>
                   | <expr> "," <typedexprlist>
    
<type> ::=  "int" | "float" | "bool" 
          | "(" <typelist> ")" "->" <type> 
          | "[" <type> "]" | "(" <typelist> ")"
          | "{" <recordtypelist> "}"
          
<recordtypelist> ::= <identifier> ":" <type> | <identifier> ":" <type> "," <recordtypelist>

<typelist> ::= <type> | <type> "," <typelist>
\end{verbatim}

\section{Sprint 9: Maths Functions}\label{sec:maths-funcs}

In this sprint we added a number of maths functions to the language, including \texttt{sin}, \texttt{cos}, \texttt{tan},
\texttt{asin}, \texttt{acos} and other, including vector operations, and added the ability to plot functions, both 
built in and user defined.
Support for complex numbers was also added in the form \texttt{1+2i}.
Due to the complication of parsing complex numbers, the type system was relaxed to allow integers to unify with any 
other number.
This was necessary as complex number parsing was often ambiguous as to whether \textit{3i + 2} should return a 
complex number directly or a sum of a complex number and an integer.
We chose the latter, as it was more intuitive and simpler to implement.

\subsection{Grammar in BNF}\label{subsec:grammar-in-bnf9}

\begin{verbatim}
<stmtlist> ::=  <stmt> 
              | <stmt> <stmtlist>
<stmt> ::=  <expr> 
          | <vardecl>
          | <assertion>
          
<vardecl> ::= "let" <identifier> "=" <expr>
            | "let" <identifier> ":" <type> "=" <expr>
            
<assertion> ::= "assert" <expr> | "assert" <expr> <string>
                

<expr> ::=  <term> 
          | <term> "+" <expr> 
          | <term> "-" <expr>
          | <term> "==" <expr>
          | <term> "!=" <expr>
          | <term> "<" <expr>
          | <term> ">" <expr>
          | <term> "<=" <expr>
          | <term> ">=" <expr>
          | <term> "&&" <expr>
          | <term> "||" <expr>
<term> ::=  <factor> 
          | <factor> "*" <term> 
          | <factor> "/" <term> 
          | <factor> "%" <term> 
<factor> ::=  <literal> 
            | "(" <expr> ")" 
            | <factor> "^" <factor>
            | <unaryop> <factor>
            | <factor> "(" <exprlist> ")"
            | <factor> "." <identifier>
            | <factor> "[" <expr> "]"
            | <factor> "." <identifier>
            | <factor> "[" <expr> ":" <expr> "]"
            | <factor> "[" <expr> ":" "]"
            | <factor> "[" ":" <expr> "]"

<unaryop> ::= "-" | "!" | "+" | <userop>

<literal> ::= <number> | <identifier> | <bool> | <list> | <lambda> | <string> | "()" | <tuple> | <record>
<string> ::= '"' <charlist> '"' | '""'
<charlist> ::= <char> | <char> <charlist>
<list> ::= "[" <exprlist> "]"
<tuple> ::= "(" <exprlist> ")"
<record> ::= "{" <recordlist> "}"
<recordlist> ::=  <identifier> "=" <expr> 
                | <identifier> "=" <expr> "," <recordlist>
                | <identifer> ":" <type> "=" <expr> 
                | <identifier> ":" <type> "=" <expr> "," <recordlist>
            
<lambda> ::=  "(" <typedexprlist> ")" "->" <expr>
            | "(" <typedexprlist> ")" ":" <type> "->" <expr>

<bool> ::= "true" | "false"
            
<number> ::= <int> | <float> | <rational> | <complex>
<int> ::= <digit> | <digit> <int>
<float> ::= <int> "." <int>
<rational> ::= <int> "/" <int>
<complex> ::= <float> "+" <float> "i" | <float> "-" <float> "i" | <float> "i"
<digit> ::= "0" | "1" | "2" | "3" | "4" | "5" | "6" | "7" | "8" | "9"

<identifier> ::= <letter> | <letter> <identifier>
<letter> ::=  "a" | "b" | "c" | "d" | "e" | "f" | "g" | "h" | "i" | "j" | "k" | "l" | "m" 
            | "n" | "o" | "p" | "q" | "r" | "s" | "t" | "u" | "v" | "w" | "x" | "y" | "z"

<exprlist> ::= <expr> | <expr> "," <exprlist>
<typedexprlist> ::=  <expr> ":" <type> 
                   | <expr> ":" <type> "," <typedexprlist>
                   | <expr> "," <typedexprlist>
    
<type> ::=  "int" | "float" | "bool" 
          | "(" <typelist> ")" "->" <type> 
          | "[" <type> "]" | "(" <typelist> ")"
          | "{" <recordtypelist> "}"
          
<recordtypelist> ::= <identifier> ":" <type> | <identifier> ":" <type> "," <recordtypelist>

<typelist> ::= <type> | <type> "," <typelist>
\end{verbatim}

\section{Sprint 10: Control flow}\label{sec:control-flow}

In this sprint we added control flow to the language, with \texttt{if} expressions and recursive bindings.
If statements were in the form \texttt{if \textit{expr} then \textit{expr} else \textit{expr}}, and required that 
the condition be a boolean and each branch be of the same type. 
The \textit{else} branch can be omitted if the \textit{then} is of type \texttt{Unit}, allowing for simple if
expressions such as \texttt{if x > 0 then print(x)}.
Additionally, if expressions are expressions, meaning they return a value, allowing for more concise code.
They are compiled as jump instructions in the bytecode compiler.

Recursive bindings were implemented using a technique called \textit{trampolining}, where a function is defined
recursively by passing the function as an argument to itself, allowing for recursion without the need for a stack.
This prevents stack overflows in the case of deep recursion, and is a common technique in functional programming.

\subsection{Grammar in BNF}\label{subsec:grammar-in-bnf10}

\begin{verbatim}
<stmtlist> ::=  <stmt> 
              | <stmt> <stmtlist>
<stmt> ::=  <expr> 
          | <vardecl>
          | <assertion>
          
<vardecl> ::= "let" <identifier> "=" <expr>
            | "let" <identifier> ":" <type> "=" <expr>
            | "let rec" <identifier> "=" <lambda>
            
<assertion> ::= "assert" <expr> | "assert" <expr> <string>

<expr> ::=  <term> 
          | <term> "+" <expr> 
          | <term> "-" <expr>
          | <term> "==" <expr>
          | <term> "!=" <expr>
          | <term> "<" <expr>
          | <term> ">" <expr>
          | <term> "<=" <expr>
          | <term> ">=" <expr>
          | <term> "&&" <expr>
          | <term> "||" <expr>
<term> ::=  <factor> 
          | <factor> "*" <term> 
          | <factor> "/" <term> 
          | <factor> "%" <term> 
<factor> ::=  <literal> 
            | "(" <expr> ")" 
            | <factor> "^" <factor>
            | <factor> "(" <exprlist> ")"
            | <factor> "." <identifier>
            | <unaryop> <factor>
            | <factor> "[" <expr> "]"
            | <factor> "[" <expr> ":" <expr> "]"
            | <factor> "[" <expr> ":" "]"
            | <factor> "[" ":" <expr> "]"
            | <if>
            | "{" <stmtlist> "}"

<unaryop> ::= "-" | "!" | "+" | <userop>

<if> ::=  "if" <expr> "then" <expr> "else" <expr>
        | "if" <expr> "then" <expr>

<literal> ::= <number> | <identifier> | <bool> | <list> | <lambda> | <string> | "()" | <tuple> | <record>
<string> ::= '"' <charlist> '"' | '""'
<charlist> ::= <char> | <char> <charlist>
<list> ::= "[" <exprlist> "]"
<tuple> ::= "(" <exprlist> ")"
<record> ::= "{" <recordlist> "}"
<recordlist> ::=  <identifier> "=" <expr> 
                | <identifier> "=" <expr> "," <recordlist>
                | <identifer> ":" <type> "=" <expr> 
                | <identifier> ":" <type> "=" <expr> "," <recordlist>
            
<lambda> ::=  "(" <typedexprlist> ")" "->" <expr>
            | "(" <typedexprlist> ")" ":" <type> "->" <expr>

<bool> ::= "true" | "false"
            
<number> ::= <int> | <float> | <rational> | <complex>
<int> ::= <digit> | <digit> <int>
<float> ::= <int> "." <int>
<rational> ::= <int> "/" <int>
<complex> ::= <float> "+" <float> "i" | <float> "-" <float> "i" | <float> "i"
<digit> ::= "0" | "1" | "2" | "3" | "4" | "5" | "6" | "7" | "8" | "9"

<identifier> ::= <letter> | <letter> <identifier>
<letter> ::=  "a" | "b" | "c" | "d" | "e" | "f" | "g" | "h" | "i" | "j" | "k" | "l" | "m" 
            | "n" | "o" | "p" | "q" | "r" | "s" | "t" | "u" | "v" | "w" | "x" | "y" | "z"

<exprlist> ::= <expr> | <expr> "," <exprlist>
<typedexprlist> ::=  <expr> ":" <type> 
                   | <expr> ":" <type> "," <typedexprlist>
                   | <expr> "," <typedexprlist>
    
<type> ::=  "int" | "float" | "bool" 
          | "(" <typelist> ")" "->" <type> 
          | "[" <type> "]" | "(" <typelist> ")"
          | "{" <recordtypelist> "}"
          
<recordtypelist> ::= <identifier> ":" <type> | <identifier> ":" <type> "," <recordtypelist>

<typelist> ::= <type> | <type> "," <typelist>
\end{verbatim}

\section{Sprint 11: Optimisation}\label{sec:optimisation1}

In this sprint we added a simple optimisation pass to the bytecode compiler, which removed unnecessary stack operations
and combined constant expressions.
Details of this are discussed in section~\ref{sec:optimisation}.
The grammar was not changed in this sprint.

\section{Sprint 12: Transpiler}\label{sec:transpiler1}

In this sprint we added the ability to transpile the bytecode to C, which could then be compiled and run as a standalone
executable.
Details of this are discussed in section~\ref{sec:transpiler}.
The grammar was not changed in this sprint.

\section{Sprint 13: Finalisation}\label{sec:finalisation1}

In this sprint we finalised the language, adding a few more functions such as matrix operations like 
\textit{transpose} and \textit{invert}, and tidied the codebase.
The grammar was not changed in this sprint.
