The language has been tested thoroughly using .Net's built in unit testing framework, NUnit.
Unit tests have been written for the lexer, parser, compiler, type inference, compiler and transpiler, and are 
within the Test project.

Additionally, tests have been written in the language itself to test the language's features.

\section{Arithmetic Expression Testing}\label{sec:arithmetic-expression-testing}

Tests can be found in table~\ref{tab:table2}.

\begin{table}[h]
    \caption{Arithmetic expression tests. Note that floating pointing values are accurate to three decimal places for the fractional part. ResE is expected result and ResA is actual result. \\}
    \begin{tabular}{|p{1.8in}|p{0.5in}|p{0.4in}|p{0.6in}|p{1.4in}|} \hline
    Expression & ResE & ResA& Pass/Fail & Action/comment \\ \hline \hline
    $5*3+(2*3-2)/2+6$ & 23 & 23 & Pass & BIDMAS \\ \hline
    $9-3-2$ & 4 & 4 & Pass & left assoc.\  \\ \hline
    $10 / 3$ & 3 & 3 & Pass & int division  \\ \hline
    $10 / 3.0$ & 3.333 & 3.333 & Pass & float division \\ \hline
    $10\%3$ & 1 & 1 & Pass & Modulus \\ \hline
    $10 - -2$ & 12 & 12 & Pass & unary minus\\ \hline
    $-2 + 10$ & 8 & * & Pass & Unary minus \\ \hline
    $3*5\verb|^|(-1+3)-2\verb|^|2*-3$ & 87 & 87 & Pass & power test \\ \hline
    $-3\verb|^|2$ & -9(*) or 9 & 9 & Pass & precedence \\ \hline
    $-7\%3$ & 2(*) or -1 & -1 & Pass & precedence (*)Python\\ \hline
    $2*3^2$ & 18 & 18 & Pass & precedence pow > mult \\ \hline
    $3*5\verb|^|(-1+3)-2\verb|^|-2*-3$ & 75.750 or 75 & 75 & Pass & Complex expression \\ \hline
    $3*5\verb|^|(-1+3)-2.0\verb|^|-2*-3$ & 75.750 & -75.750 & & \\ \hline
    $(((3*2--2)))$ & 8 & 8 & Pass & \\ \hline
    $(((3*2--2))$ & Error & Error: Expected \("\)argument or \('\)(' after call\("\) & & syntax error \\ \hline
    $-((3*5-2*3))$ & -9 & -9 & Pass & minus expression \\ \hline
        % rational numbers
    $1/2$ & 1/2 & 1/2 & Pass & Rational number \\ \hline
    $1/2 + 1/2$ & 1/1 & 1/1 & Pass & Rational number \\ \hline
    $1/2 + 1/3$ & 5/6 & 5/6 & Pass & Rational number \\ \hline
    $1/2 * 4/7$ & 2/7 & 2/7 & Pass & Rational number \\ \hline 
    $4/5 * 2/3 + 1/2 / 3/4$ & 6/5 & 6/5 & Pass & Rational number expressions \\ \hline
    $1/2 * 4 + 3 - 1/3$ & 14/3 & 14/3 & Pass & Rational number with integer \\ \hline
        % complex numbers
    $1+2i$ & 1+2i & 1+2i & Pass & Complex number \\ \hline
    $1+2i + 3+4i$ & 4+6i & 4+6i & Pass & Complex number \\ \hline
    $1+2i * 3+4i$ & 1+10i & 1+10i & Pass & Complex number \\ \hline
    \end{tabular}
    \label{tab:table2}
\end{table}

Other tests have been written in the language itself.
An example is shown below:

\begin{minted}{fsharp}
// arithmetic tests
assert 1 * 1 == 1, "Multiplication failed"
assert 1 / 1 == 1, "Division failed"
assert 1 % 1 == 0, "Modulus failed"
assert 1 ^ 1 == 1, "Exponentiation failed"
assert 1 == 1, "Equality failed"
\end{minted}

\section{Lexer testing}\label{sec:lexer-testing}

Tests for the lexer are written using NUnit and test the lexer's ability to correctly tokenise the input string.
They can be found in the source code or in the documentation.

\section{Parser testing}\label{sec:parser-testing}

Tests for the parser are written using NUnit and test the parser's ability to correctly parse the input string.
They can be found in the source code or in the documentation.

\section{Variable assignment testing}\label{sec:variable-assignment-testing}

Tests can be found in table~\ref{tab:variables}.

\begin{table}[h]
    \caption{Variable expression tests. Note that floating pointing values are accurate to three decimal places for the fractional part. ResE is expected result and ResA is actual result. \\}
    \begin{tabular}{|p{1.8in}|p{0.5in}|p{0.4in}|p{0.6in}|p{1.4in}|}
        \hline
        Expression & ResE & ResA & Pass/Fail & Action/comment \\
        \hline \hline
        $let\ x = 3$ & & & & \\
        $(2*x)-x\verb|^|2*5$ & -39 & -39 & Pass & var assign \\
        \hline
        $let\ x = 3$ & & & & \\
        $(2*x)-x\verb|^|2*5 / 2$ & -16 & & & \\
        \hline
        $let\ x = 3$ & & & & \\
        $(2*x)-x\verb|^|2*(5 / 2)$ & -12 & -12 & Pass & \\
        \hline
        $let\ x = 3$ & & & & \\
        $(2*x)-x\verb|^|2*5 / 2.0$ & -16.5 & -16.5 & Pass & Conversion \\
        \hline
        $let\ x = 3$ & & & & \\
        $(2*x)-x\verb|^|2*5\%2$ & 5 & 5 & Pass & \\
        \hline
        $let\ x = 3$ & & & & \\
        $(2*x)-x\verb|^|2*(5\%2)$ & -3 & -3 & Pass & \\
        \hline
    \end{tabular}
    \label{tab:variables}
\end{table}

More variable testing are written in the language itself.
An example is shown below:

\begin{minted}{fsharp}
// variable tests
let x = 1
assert x == 1, "Assignment failed"
assert x^2 == 1, "Assignment with expression failed"
assert x^2 + 1 == 2, "Assignment with expression failed"
\end{minted}

\section{Function testing}\label{sec:function-testing}

Tests can be found in table~\ref{tab:functions}.

\begin{table}[h]
    \caption{Function tests. Note that floating pointing values are accurate to three decimal places for the fractional part. ResE is expected result and ResA is actual result. \\}
    \begin{tabular}{|p{1.8in}|p{0.5in}|p{0.4in}|p{0.6in}|p{1.4in}|} \hline
    Expression & ResE & ResA& Pass/Fail & Action/comment \\ \hline \hline
    $let\ f = (x)\ \rightarrow x^2$ & & & & \\ 
    $f(3)$ & 9 & 9 & Pass & function \\ \hline
    $let\ f = (x)\ \rightarrow x^2$ & & & & \\ 
    $f(3) + f(2)$ & 13 & 13 & Pass & function use \\ \hline
    $let\ f = (x)\ \rightarrow \cos(x)$ & & & & \\ 
    $f(0)$ & 1.000 & 1.000 & Pass & function composition \\ \hline
    $let\ f = (x)\ \rightarrow \cos(x)$ & & & & \\ 
    $f(0) + f(0)$ & 2.000 & 2.000 & Pass & function composition \\ \hline
    $let f = () \rightarrow 5$ & & & & \\ 
    $f()$ & 5 & 5 & Pass & function no args \\ \hline
    $let f = () \rightarrow 5$ & & & & \\
    $let fa = (x) \rightarrow x * 4$ & & & & \\
    $f() + fa(3)$ & 17 & 17 & Pass & function no args \\ \hline
    $let f = () \rightarrow 5$ & & & & \\
    $let fa = (x) \rightarrow x * 4$ & & & & \\
    $fa(f())$ & 20 & 20 & Pass & function composition \\ \hline
    $let (|>) = (x, f) \rightarrow f(x)$ & & & & \\
    $5 |> (x) \rightarrow x * 4$ & 20 & 20 & Pass & operator overloading \\ \hline
    $let (|>) = (x, f) \rightarrow f(x)$ & & & & \\
    $5 |> (x) \rightarrow x * 4 |> (x) \rightarrow x + 3$ & 23 & 23 & Pass & operator overloading \\ \hline
    \end{tabular}
    \label{tab:functions}
\end{table}

Further tests are written in the language itself.
An example is shown below:

\begin{minted}{fsharp}
// function tests
let f = (x) -> x^2
assert f(3) == 9, "Function failed"
assert f(3) + f(2) == 13, "Function with expression failed"
assert f(3) + f(2) == 13, "Function with expression failed"
\end{minted}

\section{Type inference testing}\label{sec:type-inference-testing}

Type inference tests can be found in table~\ref{tab:type-inference}.

\begin{table}[h]
    \caption{Type Inference tests. ResE is expected result and ResA is actual result. \\}
    \begin{tabular}{|p{1.8in}|p{0.5in}|p{0.4in}|p{0.6in}|p{1.4in}|} \hline
    Expression & ResE & ResA& Pass/Fail & Action/comment \\ \hline \hline
    $let\ x = 3$ & int & int & Pass & int \\ \hline
    $let\ x = 3.0$ & float & float & Pass & float \\ \hline
    $let\ x = 3.0 + 2$ & float & float & Pass & addition \\ \hline
    $let\ x = 3.0 + 2.0$ & float & float & Pass & \\ \hline
    $let\ x = 3 + 2.0$ & float & float & Pass & \\ \hline
    $let\ x = 3 + 2$ & int & int & Pass & \\ \hline
    $let\ x = (() => 3)()$ & int & int & Pass & \\ \hline
    $let\ x = (() => 3.0)()$ & float & float & Pass & \\ \hline
    $let\ x = (() => [1,2])()$ & List[float] & List[float] & Pass & \\ \hline
    $let\ x = (() => {a = 3})()$ & Record[a: int] & Record[a: int] & Pass & \\ \hline
    $let\ x = (() => {a = 3, b = 4})()$ & Record[a: int, b: int] & Record[a: int, b: int] & Pass & \\ \hline
    $let\ x = 4 : float$ & float & float & Pass & Casting \\ \hline
    $let\ x = 4 : int$ & int & int & Pass & \\ \hline
    $let\ x = 4.0 : int$ & int & int & Pass & Casting \\ \hline
    $let\ x = 4.0 : float$ & float & float & Pass & \\ \hline
    \end{tabular}
    \label{tab:type-inference}
\end{table}

\section{Compound data type testing}\label{sec:compound-data-type-testing}

Tests can be found in table~\ref{tab:compound}.

\begin{table}[h]
    \caption{Compound DT tests. Note that floating pointing values are accurate to three decimal places for the fractional part. ResE is expected result and ResA is actual result. \\}
    \begin{tabular}{|p{1.8in}|p{0.5in}|p{0.4in}|p{0.6in}|p{1.4in}|} \hline
    Expression & ResE & ResA& Pass/Fail & Action/comment \\ \hline \hline
    $[1,2,3]$ & [1,2,3] & [1,2,3] & Pass & list \\ \hline
    $[1,2,3][1]$ & 2 & 2 & Pass & list index \\ \hline
    $[1,2,3][1..2]$ & [2] & [2] & Pass & list index range \\ \hline
    $[1,2,3][1..]$ & [2,3] & [2,3] & Pass & list index range \\ \hline
    $[1,2,3][..2]$ & [1] & [1] & Pass & list index range \\ \hline
    $[1..3]$ & [1,2,3] & [1,2,3] & Pass & list range \\ \hline
    $[1..3][1]$ & 2 & 2 & Pass & list range index \\ \hline
    $[1..3][1..2]$ & [2] & [2] & Pass & list range index range \\ \hline
    $\{ a = 3 \}$ & { a = 3 } & { a = 3 } & Pass & record \\ \hline
    $\{ a = 3 \}.a$ & 3 & 3 & Pass & record index \\ \hline
    $\{ a = 3 \}.b$ & Error & Error & Pass & record index \\ \hline
    $\{ a = 3, b = 4 \}.b$ & 4 & 4 & Pass & record index \\ \hline
    $\{ a = 3, b = 4 \}.a$ & 3 & 3 & Pass & record index \\ \hline
    $\{ a = 3, b = 4 \}.a + { a = 3, b = 4 }.b$ & 7 & 7 & Pass & record index \\ \hline
    \end{tabular}
    \label{tab:compound}
\end{table}

\section{Control flow testing}\label{sec:control-flow-testing}

Tests can be found in table~\ref{tab:control}.

\begin{table}[h]
    \caption{Control Flow tests. Note that floating pointing values are accurate to three decimal places for the fractional part. ResE is expected result and ResA is actual result. \\}
    \begin{tabular}{|p{1.8in}|p{0.5in}|p{0.4in}|p{0.6in}|p{1.4in}|} \hline
    Expression & ResE & ResA& Pass/Fail & Action/comment \\ \hline \hline
    $if\ 5 == 4\ then\ 3\ else\ 2$ & 2 & 2 & Pass & false \\ \hline
    $if\ 5 == 5\ then\ 3\ else\ 2$ & 3 & 3 & Pass & true \\ \hline
    $if\ 5\ then\ 3\ else\ 2$ & Error & Error & Pass & invalid type \\ \hline
    $if\ 5 == 5\ then\ 3$ & Error & Error & Pass & missing else \\ \hline
    $if\ true\ then\ print("here")$ & here & here & Pass & print no else (unit branch) \\ \hline
    $if\ false\ then\ 1\ else\ 2$ & 2 & 2 & Pass & false \\ \hline
    $let\ rec\ fact=(n)\ \rightarrow if\ n \leq 1\ then\ 1\ else\ n * fact(n-1)$ & & & & \\
    $fact(5)$ & 120 & 120 & Pass & factorial \\ \hline
    \end{tabular}
    \label{tab:control}
\end{table}

\section{GUI testing}\label{sec:gui-testing}

Tests can be found in table~\ref{tab:gui}.

\begin{table}[h]
    \caption{GUI Tests}
    \begin{tabular}{|p{1.8in}|p{0.5in}|p{0.4in}|p{0.6in}|p{1.4in}|} \hline
    Action & ResE & ResA& Pass/Fail & Action/comment \\ \hline \hline
        Click load code & Run code & Run code & Pass & Load code \\ \hline
        Click open file & Open file dialog & Open file dialog & Pass & Open file dialog \\ \hline
        Click open file & & & & \\ 
        Chose file & Opens file & Opens file & Pass & Open file \\ \hline
        Click open notebook & New notebook & New notebook & Pass & Open notebook \\ \hline
        Type in repl & & & & \\
        Click run & Run code & Run code & Pass & Output result \\ \hline
        Plot some data & Plot window & Plot window & Pass & Plot data \\ \hline
        Plot some data & & & & \\
        Run code & & & & \\
        Type in plot repl & Display plot & Display plot & Pass & Plot data \\ \hline
        Open notebook & & & & \\
        Add code & & & & \\
        Run code & Output result & Output result & Pass & Output result \\ \hline
        Open notebook & & & & \\
        Add code & & & & \\
        Run code & & & & \\
        Add code & & & & \\
        Run code with previous code use & Output result & Output result & Pass & Output result \\ \hline
        Open notebook & & & & \\
        Import file & Previous code & Previous code & Pass & Import code \\ \hline
        Open notebook & & & & \\
        Add code & & & & \\
        Save notebook & Save file & Save file & Pass & Save file \\ \hline
    \end{tabular}
    \label{tab:gui}
\end{table}

\section{Plot testing}\label{sec:plot-testing}

Tests can be found in table~\ref{tab:plot-tests}.

\begin{table}[h]
    \caption{Plot Tests}
\begin{tabular}{|p{1.8in}|p{0.5in}|p{0.4in}|p{0.6in}|p{1.4in}|} \hline
Plot & ResE & ResA& Pass/Fail & Action/comment \\ \hline \hline
$let\ x = [1..10] : [float]$ & & & & \\
$let\ y = map(x, (x) -> x^2)$ & & & & \\
$let\ data = \{ title = "Example Plot", x = x, y = y, ptype = "scatter" \}$ & & & & \\
$plot(data)$ & $f(x) = x^2$ & $f(x) = x^2$ & Pass & scatter plot \\ \hline
$let\ f = (x) -> x^2$ & & & & \\
$plotFunc("Example Plot", f)$ & $f(x) = x^2$ & $f(x) = x^2$ & Pass & plot function \\ \hline
$let\ f = (x) -> \cos(x)$ & & & & \\
$plotFunc("Example Plot", f)$ & $f(x) = \cos(x)$ & $f(x) = \cos(x)$ & Pass & plot function \\ \hline
$plotFunc("Example Plot", \cos)$ & $f(x) = \cos(x)$ & $f(x) = \cos(x)$ & Pass & plot function \\ \hline
$let\ f = (x) -> \tan(x)$ & & & & \\
$plotFunc($ & & & & \\
$"Example Plot", f)$ & $f(x) = \tan(x)$ & $f(x) = \tan(x)$ & Pass, but not infinite asymptotes & No joined asymptotes \\ \hline
$let\ f = (x) -> 1 / x$ & & & & \\
$plotFunc($ & & & & \\
$"Example Plot", f)$ & $f(x) = 1/x$ & $f(x) = 1/x$ & Pass, but not infinite asymptotes & No joined asymptotes \\ \hline
\end{tabular}
\label{tab:plot-tests}
\end{table}
