This chapter will showcase some of the maths focused algorithms used in the project.

\section{Root finding}\label{sec:root-finding}

Both the bisection and Newton-Raphson methods are implemented.

\begin{algorithm}
    \caption{Bisection method}
    \begin{algorithmic}
        \Function{bisection}{f, a, b, tol}
            \State $c \gets (a + b) / 2$
            \While{$|f(c)| > tol$}
                \If{$f(a) \cdot f(c) < 0$}
                    \State $b \gets c$
                \Else
                    \State $a \gets c$
                \EndIf
                \State $c \gets (a + b) / 2$
            \EndWhile
            \State \Return $c$
        \EndFunction
    \end{algorithmic}\label{alg:algorithm2}
\end{algorithm}

\begin{algorithm}
    \caption{Newton-Raphson method}
    \begin{algorithmic}
        \Function{newtonRaphson}{f, df, x0, tol}
            \State $x1 \gets x0 - f(x0) / df(x0)$
            \While{$|f(x1)| > tol$}
                \State $x0 \gets x1$
                \State $x1 \gets x0 - f(x0) / df(x0)$
            \EndWhile
            \State \Return $x1$
        \EndFunction
    \end{algorithmic}\label{alg:algorithm3}
\end{algorithm}

\section{Calculus}\label{sec:integration}
In the language, pure functions are represented in a symbolic expression DSL, making it easy to differentiate and 
integrate functions, as well as find tangent lines or the tailor series of a function.
The symbolic expression DSL created was inspired by the systems found at \citet{symbolicExprHask}, 
\citet{symbolicExprPython} and \citet{symbolicExprF}.

\begin{algorithm}
    \caption{Differentiation}
    \begin{algorithmic}
        \Function{differentiate}{f}
        \If{$f$ is a constant}
            \State \Return $0$
        \ElsIf{$f$ is a variable}
            \State \Return $1$
        \ElsIf{$f$ is a sum}
            \State \Return $\text{differentiate}(f_1) + \text{differentiate}(f_2)$
        \ElsIf{$f$ is a product}
            \State \Return $f_1 \cdot \text{differentiate}(f_2) + f_2 \cdot \text{differentiate}(f_1)$
        \ElsIf{$f$ is a power}
            \State \Return $n \cdot x^{n-1}$
        \ElsIf{$f$ is cos}
            \State \Return $-\sin(x)$
        \ElsIf{$f$ is sin}
            \State \Return $\cos(x)$
        \ElsIf{$f$ is tan}
            \State \Return $\sec^2(x)$
        \EndIf

        \EndFunction
    \end{algorithmic}\label{alg:algorithm4}
\end{algorithm}

\section{Vector operations}\label{sec:vector-ops}

Functions used to find the cross and dot products of two vectors are implemented, as well as standard arithmetic
operations.

\begin{algorithm}
    \caption{Cross product}
    \begin{algorithmic}
        \Function{crossProduct}{v1, v2}
            \State $x \gets v1[1] \cdot v2[3] - v1[3] \cdot v2[2]$
            \State $y \gets v1[3] \cdot v2[1] - v1[1] \cdot v2[3]$
            \State $z \gets v1[1] \cdot v2[2] - v1[2] \cdot v2[1]$
            \State \Return $[x, y, z]$
        \EndFunction
    \end{algorithmic}\label{alg:algorithm6}
\end{algorithm}

\begin{algorithm}
    \caption{Dot product}
    \begin{algorithmic}
        \Function{dotProduct}{v1, v2}
            \State $sum \gets 0$
            \For{$i \gets 1$ to $n$}
                \State $sum \gets sum + v1[i] \cdot v2[i]$
            \EndFor
            \State \Return $sum$
        \EndFunction
    \end{algorithmic}\label{alg:algorithm7}
\end{algorithm}

\section{Matrix operations}\label{sec:matrix-ops}

The language has built-in functions for finding the transpose, determinant and inverse of a matrix.

\begin{algorithm}
    \caption{Transpose}
    \begin{algorithmic}
        \Function{transpose}{m}
            \State $rows \gets \text{length}(m)$
            \State $cols \gets \text{length}(m[1])$
            \State $trans \gets \text{emptyMatrix}(cols, rows)$
            \For{$i \gets 1$ to $rows$}
                \For{$j \gets 1$ to $cols$}
                    \State $trans[j][i] \gets m[i][j]$
                \EndFor
            \EndFor
            \State \Return $trans$
        \EndFunction
    \end{algorithmic}\label{alg:algorithm8}
\end{algorithm}

\begin{algorithm}
    \caption{Determinant}
    \begin{algorithmic}
        \Function{determinant}{m}
            \State $rows \gets \text{length}(m)$
            \State $cols \gets \text{length}(m[1])$
            \If{$rows \neq cols$}
                \State \Return $0$
            \EndIf
            \If{$rows = 2$}
                \State \Return $m[1][1] \cdot m[2][2] - m[1][2] \cdot m[2][1]$
            \EndIf
            \State $\det \gets 0$
            \For{$i \gets 1$ to $rows$}
                \State $\det \gets \det + m[1][i] \cdot \text{cofactor}(m, 1, i)$
            \EndFor
            \State \Return $\det$
        \EndFunction
    \end{algorithmic}\label{alg:algorithm9}
\end{algorithm}

\begin{algorithm}
    \caption{Inverse}
    \begin{algorithmic}
        \Function{inverse}{m}
            \State $rows \gets \text{length}(m)$
            \State $cols \gets \text{length}(m[1])$
            \State $\det \gets \text{determinant}(m)$
            \If{$\det = 0$}
                \State \Return \text{error}
            \EndIf
            \State $inv \gets \text{emptyMatrix}(rows, cols)$
            \For{$i \gets 1$ to $rows$}
                \For{$j \gets 1$ to $cols$}
                    \State $inv[i][j] \gets \text{cofactor}(m, i, j) / \det$
                \EndFor
            \EndFor
            \State \Return $inv$
        \EndFunction
    \end{algorithmic}\label{alg:algorithm10}
\end{algorithm}