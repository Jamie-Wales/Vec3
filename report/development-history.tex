\section{Sprint 1: Basic expressions}\label{sec:sprint-1:-basic-expressions}

This sprint focused on implementing a lexer and parser for the language, with precedence rules for the arithmetic 
operators, parsed with Pratt parsing.

\subsection{Grammar in BNF}\label{subsec:grammar-in-bnf1}

\begin{verbatim}
<expr> ::= <term> | <term> "+" <expr> | <term> "-" <expr>
<term> ::= <factor> | <factor> "*" <term> | <factor> "/" <term>
<factor> ::= <number> | "(" <expr> ")"
<number> ::= <int> | <float>
<int> ::= <digit> | <digit> <int>
<float> ::= <int> "." <int>
<digit> ::= "0" | "1" | "2" | "3" | "4" | "5" | "6" | "7" | "8" | "9"
\end{verbatim}

\section{Sprint 2: Variable assignment}\label{sec:variable-assignment}

In this sprint we added variable assignment to the language, with the ability to bind an expression to a variable name.

\subsection{Grammar in BNF}\label{subsec:grammar-in-bnf2}

\begin{verbatim}
<stmtlist> ::= <stmt> | <stmt> <stmtlist>
<stmt> ::= <expr> | "let" <identifier> "=" <expr>

<expr> ::= <term> | <term> "+" <expr> | <term> "-" <expr>
<term> ::= <factor> | <factor> "*" <term> | <factor> "/" <term> | <factor> "%" <term>
<factor> ::= <number> | <identifier> | "(" <expr> ")" | <factor> "^" <factor>

<number> ::= <int> | <float>

<int> ::= <digit> | <digit> <int>
<float> ::= <int> "." <int>
<digit> ::= "0" | "1" | "2" | "3" | "4" | "5" | "6" | "7" | "8" | "9"

<identifier> ::= <letter> | <letter> <identifier>
<letter> ::= "a" | "b" | "c" | "d" | "e" | "f" | "g" | "h" | "i" | "j" | "k" | "l" | "m" | "n" | "o" | "p" | "q" | "r" | "s" | "t" | "u" | "v" | "w" | "x" | "y" | "z"
\end{verbatim}

\section{Sprint 3: Interpreter}\label{sec:interpreter}

In this sprint a basic interpreter was implemented, with the ability to evaluate expressions and variable bindings.

We used a simple environment to store variable bindings, and a recursive evaluation function to evaluate expressions.

It was a REPL style interpreter, where the last expression of a statement list was evaluated and printed.

\subsection{Grammar in BNF}\label{subsec:grammar-in-bnf3}

\begin{verbatim}

\end{verbatim}

\section{Sprint 4: Functions}\label{sec:functions}

In this sprint we added the ability to define and call functions, with a simple lambda syntax of the form 
\texttt{(\textit{args}) -> \textit{expr}} and function calls of the form \texttt{\textit{funcName / lambda}(\textit{args})}.

Call by value semantics were used, with a new environment created for each function call.

\subsection{Grammar in BNF}\label{subsec:grammar-in-bnf4}

\begin{verbatim}
<stmtlist> ::= <stmt> | <stmt> <stmtlist>
<stmt> ::= <expr> | "let" <identifier> "=" <expr>
<expr> ::= <term> | <term> "+" <expr> | <term> "-" <expr>
<term> ::= <factor> | <factor> "*" <term> | <factor> "/" <term> | <factor> "%" <term>
<factor> ::= <number> | <identifier> | "(" <expr> ")" | <factor> "^" <factor> | <identifier> "(" <exprlist> ")" | <lambda>
<lambda> ::= "(" <exprlist> ")" "->" <expr> 

<number> ::= <int> | <float>
<int> ::= <digit> | <digit> <int>
<float> ::= <int> "." <int>

<digit> ::= "0" | "1" | "2" | "3" | "4" | "5" | "6" | "7" | "8" | "9"
<identifier> ::= <letter> | <letter> <identifier>
<letter> ::= "a" | "b" | "c" | "d" | "e" | "f" | "g" | "h" | "i" | "j" | "k" | "l" | "m" | "n" | "o" | "p" | "q" | "r" | "s" | "t" | "u" | "v" | "w" | "x" | "y" | "z"
<exprlist> ::= <expr> | <expr> "," <exprlist>

\end{verbatim}

\section{Sprint 5: Static type checking}\label{sec:static-type-checking}

In this sprint we added static type checking to the language, with a simple type inference system based on 
Hindley-Milner.

The concept of types was introduced, with the types \texttt{Int}, \texttt{Float}, \texttt{Bool}, \texttt{Function} and
\texttt{Never}.

\subsection{Grammar in BNF}\label{subsec:grammar-in-bnf5}

\begin{verbatim}
\end{verbatim}

\section{Sprint 6: Bytecode}\label{sec:bytecode}

In this sprint the interpreter was rewritten to use a bytecode interpreter, with a stack based virtual machine as 
well as a simple bytecode compiler, allowing for more efficient evaluation of expressions.

\subsection{Grammar in BNF}\label{subsec:grammar-in-bnf6}

\begin{verbatim}
\end{verbatim}

\section{Sprint 7: GUI}\label{sec:gui}

A simple GUI was developed in order to allow easier testing of the language, with a text box for input and output and a 
decompiler output for debugging.

\subsection{Grammar in BNF}\label{subsec:grammar-in-bnf7}

\begin{verbatim}
\end{verbatim}

\section{Sprint 8: Plotting}\label{sec:plotting1}

In this sprint we added the ability to plot lists of points, with a simple plotting function that took a list of
\textit{x} coordinates and a list of \textit{y} coordinates and plotted them on a graph using \citet{scottPlot}.

\subsection{Grammar in BNF}\label{subsec:grammar-in-bnf8}

\begin{verbatim}
\end{verbatim}

\section{Sprint 9: Maths Functions}\label{sec:maths-funcs}

In this sprint we added a number of maths functions to the language, including \texttt{sin}, \texttt{cos}, \texttt{tan},
\texttt{asin}, \texttt{acos} and others, and added the ability to plot functions, both built in and user defined.

\subsection{Grammar in BNF}\label{subsec:grammar-in-bnf9}

\begin{verbatim}
\end{verbatim}

\section{Sprint 9: Optimisation}\label{sec:optimisation1}

In this sprint we added a simple optimisation pass to the bytecode compiler, which removed unnecessary stack operations
and combined constant expressions.

\subsection{Grammar in BNF}\label{subsec:grammar-in-bnf10}

\begin{verbatim}
\end{verbatim}