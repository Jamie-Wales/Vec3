
\documentclass[a4paper, oneside, 11pt]{report}
\usepackage{minted}
\usepackage{graphicx}
\usepackage{subfig}
\usepackage{syntax}
\usepackage{epsfig,pifont,float,multirow,amsmath,amssymb}
\usepackage{hyperref}
\usepackage{enumitem}
\usepackage{pdfpages}

\setlist{nolistsep}

\newcommand{\mc}{\multicolumn{1}{c|}}
\newcommand{\mb}{\mathbf}
\newcommand{\mi}{\mathit}
\newcommand{\oa}{\overrightarrow}
\newcommand{\bs}{\boldsymbol}
\newcommand{\ra}{\rightarrow}
\newcommand{\la}{\leftarrow}
\usepackage{algorithm}
\usepackage{natbib}
\usepackage{algpseudocode}
\topmargin = 0pt
\voffset = -80pt
\oddsidemargin = 15pt
\textwidth = 425pt
\textheight = 750pt
\begin{document}

\begin{titlepage}
\begin{center}
\rule{12cm}{1mm} \\
\vspace{1cm}
{\large  CMP-6048A Advanced Programming} %Delete as appropriate
\vspace{7.5cm}
\\{\Large Project Report - 13 January 2025}
\vspace{1.5cm}
\\{\LARGE Vec3 Maths Bytecode Interpreter} % You can add to this title of modify it if you wish
\vspace{1.0cm}
\\{\Large Group members: \\ Jamie Wales, Jacob Edwards.\ }
\vspace{10.0cm}
\\{\large School of Computing Sciences, University of East Anglia}
\\ \rule{12cm}{0.5mm}
\\ \hspace{8.5cm} {\large Version 2.0}
\end{center}
\end{titlepage}


\setcounter{page}{1}

\begin{abstract}
Vec3 is a bytecode interpreted maths language complete with a GUI, plotting, full static type inference and various 
maths functions.
The language is designed to be simple to use and understand, with a focus on strict mathematical expressions and 
plotting, but with more powerful constructs such as recursive bindings, first class functions and static type inference.
The language and GUI are written in F\#, using Avalonia\citep{avalonia} for the GUI and ScottPlot\citep{scottPlot} for plotting.
    
The language is compiled to a custom bytecode, which is then interpreted by a virtual machine. 
The language also has the
ability to transpile to C, which can then be compiled and run as a standalone executable, allowing for faster execution of
the code.
    
It is a functional-style language, with a focus on immutability, recursion, expressions and correctness.

Some of the features of the language include:

\begin{itemize}
    \item Plotting of functions and data
    \item Recursive bindings
    \item First class functions
    \item Static type inference
    \item Strongly typed vector and matrix types
    \item Transpilation to C
    \item Async functions
    \item A GUI
    \item Lots of maths utilities, many of which are implemented in the language itself
\end{itemize}
    
We used an agile approach during development, where we would work on features in small iterations, with regular meetings
and discussions to ensure we were on track.
\end{abstract}

\chapter{Introduction}\label{ch:intro}

\section{Project statement}\label{sec:project-statement}

Vec3 is a bytecode interpreted maths language complete with a GUI, plotting, full static type inference and various
maths functions.
The language is designed to be simple to use and understand, with a focus on strict mathematical expressions and plotting,
but with more powerful constructs such as recursive bindings, first class functions and static type inference.
The language and GUI are written in F\#, using Avalonia\citep{avalonia} for the GUI and ScottPlot\citep{scottPlot} for plotting.

It also has the ability to transpile to C, which can then be compiled and run as a standalone executable, allowing for
faster execution of the code.

\section{Aims and objectives}\label{sec:aims-and-objectives}

The aim of the project is to create an easy to use maths language with a focus on plotting and mathematical
expressions, but with more powerful constructs such as recursive bindings, first class functions and static type 
inference, and ensuring as many features, both optional and mandatory, are implemented as possible and to a high 
standard.

\begin{table}[h]
    \caption{MoSCoW}
    \begin{center}
        \begin{tabular}{|p{1in}|p{2in}|p{2.5in}|} \hline
        Priority & Task & Comments \\ \hline \hline
        \multirow{3}{1in}{Must}
        & Arithmetic operations & $+, -, /, *, \%, ^$ \\ \cline{2-3}
        & Plotting & Plotting of functions and data \\ \cline{2-3}
        & Functions & User defined functions \\ \cline{2-3}
        & GUI & A GUI for the language with a text editor \\ \cline{2-3}
        & Maths tools & Lots of maths utilities (cos, etc) \\ \cline{2-3}
        \multirow{3}{1in}{Should}
        & Rational numbers & Rational number support \\ \cline{2-3}
        & Complex numbers & Complex number support \\ \cline{2-3}
        & Static typing & Static typing and analysis \\ \cline{2-3}
        & First class functions & First class functions \\ \cline{2-3}
        & Control flow & If statements, loops \\ \cline{2-3}
        & Compound data types & Vectors, matrices, records \\ \cline{2-3}
        \multirow{3}{1in}{Could}
        & Compilation & ByteCode compilation and interpretation \\ \cline{2-3}
        & Transpilation & Transpilation to C \\ \cline{2-3}
        & Async & Async functions \\ \cline{2-3}
        & Static type inference & Hindler-Miller type inference \\ \cline{2-3}
        & Error handling & Error handling \\ \cline{2-3}
        & Recursion & Recursive bindings \\ \cline{2-3}
        & Drawing & Drawing on the GUI \\ \cline{2-3}
        & Importing & Importing of external libraries \\ \cline{2-3}
        \multirow{3}{1in}{Should not}
        & Compilation & Compilation directly to ASM \\ \cline{2-3}
        & Transpilation & Transpilation to other languages \\ \cline{2-3}
        & GUI & A web-based GUI \\ \cline{2-3}
        & FFI & Foreign function interface \\ \cline{2-3}
        \hline
        \end{tabular}
        \label{Table1}
    \end{center}
\end{table}

\chapter{Background}\label{ch:background}

\section{The Task}\label{sec:the-task}

The task was to create a maths language with a focus on plotting and mathematical expressions, as well as an 
integrated GUI\@, with possible support for rational and complex numbers, control flow and functions.

\section{Other languages}\label{sec:other-languages}

\chapter{Development History}\label{ch:devhist}

\section{Sprint 1: Basic expressions}\label{sec:sprint-1:-basic-expressions}

This sprint focused on implementing a lexer and parser for the language, with precedence rules for the arithmetic 
operators, parsed with Pratt parsing.

\subsection{Grammar in BNF}\label{subsec:grammar-in-bnf1}

\begin{verbatim}
<expr> ::= <term> | <term> '+' <expr> | <term> "-" <expr>
<term> ::= <factor> | <factor> "*" <term> | <factor> "/" <term>
<factor> ::= <number> | "(" <expr> ")"
<number> ::= <int> | <float>
<int> ::= <digit> | <digit> <int>
<float> ::= <int> "." <int>
<digit> ::= "0" | "1" | "2" | "3" | "4" | "5" | "6" | "7" | "8" | "9"
\end{verbatim}

\section{Sprint 2: Variable assignment}\label{sec:variable-assignment}

In this sprint we added variable assignment to the parser, with the ability to bind an expression to a variable 
name, as well as a few new operators such as \textit{==} for equality and \textit{\%} for modulo, as well as 
unary operators.

\subsection{Grammar in BNF}\label{subsec:grammar-in-bnf2}

\begin{verbatim}
<stmtlist> ::=  <stmt> 
              | <stmt> <stmtlist>
<stmt> ::=  <expr> 
          | "let" <identifier> "=" <expr>

<expr> ::=  <term> 
          | <term> "+" <expr> 
          | <term> "-" <expr>
          | <term> "==" <expr>
          | <term> "!=" <expr>
          | <term> "<" <expr>
          | <term> ">" <expr>
          | <term> "<=" <expr>
          | <term> ">=" <expr>
          | <term> "&&" <expr>
          | <term> "||" <expr>
<term> ::=  <factor> 
          | <factor> "*" <term> 
          | <factor> "/" <term> 
          | <factor> "%" <term> 
<factor> ::=  <number> 
            | <identifier> 
            | <unaryop> <factor>
            | "(" <expr> ")" 
            | <factor> "^" <factor>

<unaryop> ::= "-" | "!" | "+" | <userop>

<number> ::= <int> | <float>
<int> ::= <digit> | <digit> <int>
<float> ::= <int> "." <int>
<digit> ::= "0" | "1" | "2" | "3" | "4" | "5" | "6" | "7" | "8" | "9"

<identifier> ::= <letter> | <letter> <identifier>
<letter> ::=  "a" | "b" | "c" | "d" | "e" | "f" | "g" | "h" | "i" | "j" | "k" | "l" | "m" 
            | "n" | "o" | "p" | "q" | "r" | "s" | "t" | "u" | "v" | "w" | "x" | "y" | "z"
\end{verbatim}

\section{Sprint 3: Interpreter}\label{sec:interpreter}

In this sprint a basic interpreter was implemented, with the ability to evaluate expressions and variable bindings.
We used a simple environment to store variable bindings (a map of \textit{string} name to \textit{expr}), and a 
recursive evaluation function to evaluate expressions.
It was a REPL style interpreter, where the last expression of a statement list was evaluated and printed, and ran through
the command line.
We did not change the grammar in this sprint.

\section{Sprint 4: Functions}\label{sec:functions}

In this sprint we added the ability to define and call functions, with a simple lambda syntax of the form 
\texttt{(\textit{args}) -> \textit{expr}} and function calls of the form \texttt{\textit{funcName / lambda}(\textit{args})}.
Call by value semantics were used, with a new environment created for each function call, consisting of the arguments
bound to the parameter names and the parent environment.

We also added an \textit{assert} statement, allowing for simple tests to be written in the language and support for 
rational numbers.

\subsection{Grammar in BNF}\label{subsec:grammar-in-bnf4}

\begin{verbatim}
<stmtlist> ::=  <stmt> 
              | <stmt> <stmtlist>
<stmt> ::=  <expr> 
          | "let" <identifier> "=" <expr>

<expr> ::=  <term> 
          | <term> "+" <expr> 
          | <term> "-" <expr>
          | <term> "==" <expr>
          | <term> "!=" <expr>
          | <term> "<" <expr>
          | <term> ">" <expr>
          | <term> "<=" <expr>
          | <term> ">=" <expr>
          | <term> "&&" <expr>
          | <term> "||" <expr>
<term> ::=  <factor> 
          | <factor> "*" <term> 
          | <factor> "/" <term> 
          | <factor> "%" <term> 
<factor> ::=  <number> 
            | <unaryop> <factor>
            | <identifier> 
            | "(" <expr> ")" 
            | <factor> "^" <factor>
            | <factor> "(" <exprlist> ")"
            | <lambda>
<lambda> ::= "(" <exprlist> ")" "->" <expr>

<unaryop> ::= "-" | "!" | "+" | <userop>
            
<number> ::= <int> | <float> | <rational>
<int> ::= <digit> | <digit> <int>
<float> ::= <int> "." <int>
<rational> ::= <int> "/" <int>
<digit> ::= "0" | "1" | "2" | "3" | "4" | "5" | "6" | "7" | "8" | "9"

<identifier> ::= <letter> | <letter> <identifier>
<letter> ::=  "a" | "b" | "c" | "d" | "e" | "f" | "g" | "h" | "i" | "j" | "k" | "l" | "m" 
            | "n" | "o" | "p" | "q" | "r" | "s" | "t" | "u" | "v" | "w" | "x" | "y" | "z"
            
<exprlist> ::= <expr> | <expr> "," <exprlist>

\end{verbatim}

\section{Sprint 5: Static type checking}\label{sec:static-type-checking}

In this sprint we added static type checking to the language, with a simple type inference system based on 
Hindley-Milner.

The concept of types was introduced, with the types \texttt{Int}, \texttt{Float}, \texttt{Bool}, \texttt{Function} and
\texttt{Never}.

\subsection{Grammar in BNF}\label{subsec:grammar-in-bnf5}

\begin{verbatim}
<stmtlist> ::=  <stmt> 
              | <stmt> <stmtlist>
<stmt> ::=  <expr> 
          | "let" <identifier> "=" <expr>
          | "let" <identifier> ":" <type> "=" <expr>

<expr> ::=  <term> 
          | <term> "+" <expr> 
          | <term> "-" <expr>
          | <term> "==" <expr>
          | <term> "!=" <expr>
          | <term> "<" <expr>
          | <term> ">" <expr>
          | <term> "<=" <expr>
          | <term> ">=" <expr>
          | <term> "&&" <expr>
          | <term> "||" <expr>
<term> ::=  <factor> 
          | <factor> "*" <term> 
          | <factor> "/" <term> 
          | <factor> "%" <term> 
<factor> ::=  <number> 
            | <identifier> 
            | <unaryop> <factor>
            | "(" <expr> ")" 
            | <factor> "^" <factor>
            | <factor> "(" <exprlist> ")"
            | <bool>
            | <lambda>
<lambda> ::=  "(" <typedexprlist> ")" "->" <expr>
            | "(" <typedexprlist> ")" ":" <type> "->" <expr>


<unaryop> ::= "-" | "!" | "+" | <userop>

<bool> ::= "true" | "false"
            
<number> ::= <int> | <float> | <rational>
<int> ::= <digit> | <digit> <int>
<float> ::= <int> "." <int>
<rational> ::= <int> "/" <int>
<digit> ::= "0" | "1" | "2" | "3" | "4" | "5" | "6" | "7" | "8" | "9"

<identifier> ::= <letter> | <letter> <identifier>
<letter> ::=  "a" | "b" | "c" | "d" | "e" | "f" | "g" | "h" | "i" | "j" | "k" | "l" | "m" 
            | "n" | "o" | "p" | "q" | "r" | "s" | "t" | "u" | "v" | "w" | "x" | "y" | "z"
            
<exprlist> ::= <expr> | <expr> "," <exprlist>
<typedexprlist> ::=  <expr> ":" <type> 
                   | <expr> ":" <type> "," <typedexprlist>
                   | <expr> "," <typedexprlist>
    
<type> ::= "int" | "float" | "bool" | "(" <typelist> ")" "->" <type>

<typelist> ::= <type> | <type> "," <typelist>
\end{verbatim}

\section{Sprint 6: Bytecode}\label{sec:bytecode}

In this sprint the interpreter was rewritten to use a bytecode interpreter, with a stack based virtual machine as 
well as a simple bytecode compiler, allowing for more efficient evaluation of expressions.
The grammar was not changed in this sprint.

\section{Sprint 7: GUI}\label{sec:gui}

A simple GUI was developed in order to allow easier testing of the language, with a text box for input and output and a 
decompiler output for debugging.
The GUI was written in F\# using Avalonia\citep{avalonia}.
We did not change the grammar in this sprint.

\section{Sprint 8: Plotting}\label{sec:plotting1}

In this sprint we added the ability to plot lists of points, with a simple plotting function that took a list of
\textit{x} coordinates and a list of \textit{y} coordinates and plotted them on a graph using \citet{scottPlot}.
Naturally, we had to add a new type, \texttt{List}, to the language, and as an extension of this, we added the ability to
define lists using the syntax \texttt{[1, 2, 3, 4]}.
Other compound data types such as tuples and records were also added.

\subsection{Grammar in BNF}\label{subsec:grammar-in-bnf8}

\begin{verbatim}
<stmtlist> ::=  <stmt> 
              | <stmt> <stmtlist>
<stmt> ::=  <expr> 
          | <vardecl>
          | <assertion>
          
<vardecl> ::= "let" <identifier> "=" <expr>
            | "let" <identifier> ":" <type> "=" <expr>
            
<assertion> ::= "assert" <expr> | "assert" <expr> <string>
                

<expr> ::=  <term> 
          | <term> "+" <expr> 
          | <term> "-" <expr>
          | <term> "==" <expr>
          | <term> "!=" <expr>
          | <term> "<" <expr>
          | <term> ">" <expr>
          | <term> "<=" <expr>
          | <term> ">=" <expr>
          | <term> "&&" <expr>
          | <term> "||" <expr>
<term> ::=  <factor> 
          | <factor> "*" <term> 
          | <factor> "/" <term> 
          | <factor> "%" <term> 
<factor> ::=  <literal> 
            | "(" <expr> ")" 
            | <factor> "^" <factor>
            | <unaryop> <factor>
            | <factor> "(" <exprlist> ")"
            | <factor> "." <identifier>
            | <factor> "[" <expr> "]"
            | <factor> "." <identifier>
            | <factor> "[" <expr> ":" <expr> "]"
            | <factor> "[" <expr> ":" "]"
            | <factor> "[" ":" <expr> "]"

<unaryop> ::= "-" | "!" | "+" | <userop>

<literal> ::= <number> | <identifier> | <bool> | <list> | <lambda> | <string> | "()" | <tuple> | <record>
<string> ::= '"' <charlist> '"' | '""'
<charlist> ::= <char> | <char> <charlist>
<list> ::= "[" <exprlist> "]"
<tuple> ::= "(" <exprlist> ")"
<record> ::= "{" <recordlist> "}"
<recordlist> ::=  <identifier> "=" <expr> 
                | <identifier> "=" <expr> "," <recordlist>
                | <identifer> ":" <type> "=" <expr> 
                | <identifier> ":" <type> "=" <expr> "," <recordlist>
            
<lambda> ::=  "(" <typedexprlist> ")" "->" <expr>
            | "(" <typedexprlist> ")" ":" <type> "->" <expr>

<bool> ::= "true" | "false"
            
<number> ::= <int> | <float> | <rational>
<int> ::= <digit> | <digit> <int>
<float> ::= <int> "." <int>
<rational> ::= <int> "/" <int>
<digit> ::= "0" | "1" | "2" | "3" | "4" | "5" | "6" | "7" | "8" | "9"

<identifier> ::= <letter> | <letter> <identifier>
<letter> ::=  "a" | "b" | "c" | "d" | "e" | "f" | "g" | "h" | "i" | "j" | "k" | "l" | "m" 
            | "n" | "o" | "p" | "q" | "r" | "s" | "t" | "u" | "v" | "w" | "x" | "y" | "z"

<exprlist> ::= <expr> | <expr> "," <exprlist>
<typedexprlist> ::=  <expr> ":" <type> 
                   | <expr> ":" <type> "," <typedexprlist>
                   | <expr> "," <typedexprlist>
    
<type> ::=  "int" | "float" | "bool" 
          | "(" <typelist> ")" "->" <type> 
          | "[" <type> "]" | "(" <typelist> ")"
          | "{" <recordtypelist> "}"
          
<recordtypelist> ::= <identifier> ":" <type> | <identifier> ":" <type> "," <recordtypelist>

<typelist> ::= <type> | <type> "," <typelist>
\end{verbatim}

\section{Sprint 9: Maths Functions}\label{sec:maths-funcs}

In this sprint we added a number of maths functions to the language, including \texttt{sin}, \texttt{cos}, \texttt{tan},
\texttt{asin}, \texttt{acos} and other, including vector operations, and added the ability to plot functions, both 
built in and user defined.
Support for complex numbers was also added.

\subsection{Grammar in BNF}\label{subsec:grammar-in-bnf9}

\begin{verbatim}
<stmtlist> ::=  <stmt> 
              | <stmt> <stmtlist>
<stmt> ::=  <expr> 
          | <vardecl>
          | <assertion>
          
<vardecl> ::= "let" <identifier> "=" <expr>
            | "let" <identifier> ":" <type> "=" <expr>
            
<assertion> ::= "assert" <expr> | "assert" <expr> <string>
                

<expr> ::=  <term> 
          | <term> "+" <expr> 
          | <term> "-" <expr>
          | <term> "==" <expr>
          | <term> "!=" <expr>
          | <term> "<" <expr>
          | <term> ">" <expr>
          | <term> "<=" <expr>
          | <term> ">=" <expr>
          | <term> "&&" <expr>
          | <term> "||" <expr>
<term> ::=  <factor> 
          | <factor> "*" <term> 
          | <factor> "/" <term> 
          | <factor> "%" <term> 
<factor> ::=  <literal> 
            | "(" <expr> ")" 
            | <factor> "^" <factor>
            | <unaryop> <factor>
            | <factor> "(" <exprlist> ")"
            | <factor> "." <identifier>
            | <factor> "[" <expr> "]"
            | <factor> "." <identifier>
            | <factor> "[" <expr> ":" <expr> "]"
            | <factor> "[" <expr> ":" "]"
            | <factor> "[" ":" <expr> "]"

<unaryop> ::= "-" | "!" | "+" | <userop>

<literal> ::= <number> | <identifier> | <bool> | <list> | <lambda> | <string> | "()" | <tuple> | <record>
<string> ::= '"' <charlist> '"' | '""'
<charlist> ::= <char> | <char> <charlist>
<list> ::= "[" <exprlist> "]"
<tuple> ::= "(" <exprlist> ")"
<record> ::= "{" <recordlist> "}"
<recordlist> ::=  <identifier> "=" <expr> 
                | <identifier> "=" <expr> "," <recordlist>
                | <identifer> ":" <type> "=" <expr> 
                | <identifier> ":" <type> "=" <expr> "," <recordlist>
            
<lambda> ::=  "(" <typedexprlist> ")" "->" <expr>
            | "(" <typedexprlist> ")" ":" <type> "->" <expr>

<bool> ::= "true" | "false"
            
<number> ::= <int> | <float> | <rational> | <complex>
<int> ::= <digit> | <digit> <int>
<float> ::= <int> "." <int>
<rational> ::= <int> "/" <int>
<complex> ::= <float> "+" <float> "i" | <float> "-" <float> "i" | <float> "i"
<digit> ::= "0" | "1" | "2" | "3" | "4" | "5" | "6" | "7" | "8" | "9"

<identifier> ::= <letter> | <letter> <identifier>
<letter> ::=  "a" | "b" | "c" | "d" | "e" | "f" | "g" | "h" | "i" | "j" | "k" | "l" | "m" 
            | "n" | "o" | "p" | "q" | "r" | "s" | "t" | "u" | "v" | "w" | "x" | "y" | "z"

<exprlist> ::= <expr> | <expr> "," <exprlist>
<typedexprlist> ::=  <expr> ":" <type> 
                   | <expr> ":" <type> "," <typedexprlist>
                   | <expr> "," <typedexprlist>
    
<type> ::=  "int" | "float" | "bool" 
          | "(" <typelist> ")" "->" <type> 
          | "[" <type> "]" | "(" <typelist> ")"
          | "{" <recordtypelist> "}"
          
<recordtypelist> ::= <identifier> ":" <type> | <identifier> ":" <type> "," <recordtypelist>

<typelist> ::= <type> | <type> "," <typelist>
\end{verbatim}

\section{Sprint 10: Control flow}\label{sec:control-flow}

In this sprint we added control flow to the language, with \texttt{if} expressions and recursive bindings.

\subsection{Grammar in BNF}\label{subsec:grammar-in-bnf10}

\begin{verbatim}
<stmtlist> ::=  <stmt> 
              | <stmt> <stmtlist>
<stmt> ::=  <expr> 
          | <vardecl>
          | <assertion>
          
<vardecl> ::= "let" <identifier> "=" <expr>
            | "let" <identifier> ":" <type> "=" <expr>
            | "let rec" <identifier> "=" <lambda>
            
<assertion> ::= "assert" <expr> | "assert" <expr> <string>

<expr> ::=  <term> 
          | <term> "+" <expr> 
          | <term> "-" <expr>
          | <term> "==" <expr>
          | <term> "!=" <expr>
          | <term> "<" <expr>
          | <term> ">" <expr>
          | <term> "<=" <expr>
          | <term> ">=" <expr>
          | <term> "&&" <expr>
          | <term> "||" <expr>
<term> ::=  <factor> 
          | <factor> "*" <term> 
          | <factor> "/" <term> 
          | <factor> "%" <term> 
<factor> ::=  <literal> 
            | "(" <expr> ")" 
            | <factor> "^" <factor>
            | <factor> "(" <exprlist> ")"
            | <factor> "." <identifier>
            | <unaryop> <factor>
            | <factor> "[" <expr> "]"
            | <factor> "[" <expr> ":" <expr> "]"
            | <factor> "[" <expr> ":" "]"
            | <factor> "[" ":" <expr> "]"
            | <if>
            | "{" <stmtlist> "}"

<unaryop> ::= "-" | "!" | "+" | <userop>

<if> ::=  "if" <expr> "then" <expr> "else" <expr>
        | "if" <expr> "then" <expr>

<literal> ::= <number> | <identifier> | <bool> | <list> | <lambda> | <string> | "()" | <tuple> | <record>
<string> ::= '"' <charlist> '"' | '""'
<charlist> ::= <char> | <char> <charlist>
<list> ::= "[" <exprlist> "]"
<tuple> ::= "(" <exprlist> ")"
<record> ::= "{" <recordlist> "}"
<recordlist> ::=  <identifier> "=" <expr> 
                | <identifier> "=" <expr> "," <recordlist>
                | <identifer> ":" <type> "=" <expr> 
                | <identifier> ":" <type> "=" <expr> "," <recordlist>
            
<lambda> ::=  "(" <typedexprlist> ")" "->" <expr>
            | "(" <typedexprlist> ")" ":" <type> "->" <expr>

<bool> ::= "true" | "false"
            
<number> ::= <int> | <float> | <rational> | <complex>
<int> ::= <digit> | <digit> <int>
<float> ::= <int> "." <int>
<rational> ::= <int> "/" <int>
<complex> ::= <float> "+" <float> "i" | <float> "-" <float> "i" | <float> "i"
<digit> ::= "0" | "1" | "2" | "3" | "4" | "5" | "6" | "7" | "8" | "9"

<identifier> ::= <letter> | <letter> <identifier>
<letter> ::=  "a" | "b" | "c" | "d" | "e" | "f" | "g" | "h" | "i" | "j" | "k" | "l" | "m" 
            | "n" | "o" | "p" | "q" | "r" | "s" | "t" | "u" | "v" | "w" | "x" | "y" | "z"

<exprlist> ::= <expr> | <expr> "," <exprlist>
<typedexprlist> ::=  <expr> ":" <type> 
                   | <expr> ":" <type> "," <typedexprlist>
                   | <expr> "," <typedexprlist>
    
<type> ::=  "int" | "float" | "bool" 
          | "(" <typelist> ")" "->" <type> 
          | "[" <type> "]" | "(" <typelist> ")"
          | "{" <recordtypelist> "}"
          
<recordtypelist> ::= <identifier> ":" <type> | <identifier> ":" <type> "," <recordtypelist>

<typelist> ::= <type> | <type> "," <typelist>
\end{verbatim}

\section{Sprint 11: Optimisation}\label{sec:optimisation1}

In this sprint we added a simple optimisation pass to the bytecode compiler, which removed unnecessary stack operations
and combined constant expressions.
The grammar was not changed in this sprint.

\section{Sprint 12: Transpiler}\label{sec:transpiler1}

In this sprint we added the ability to transpile the bytecode to C, which could then be compiled and run as a standalone
executable.
The grammar was not changed in this sprint.



\chapter{Final deliverable}\label{ch:impl}

\section{Final BNF}\label{sec:final-bnf}

\begin{verbatim}
<stmtlist> ::=  <stmt> 
              | <stmt> <stmtlist>
<stmt> ::=  <expr> 
          | <vardecl>
          | <assertion>

<vardecl> ::= "let" <identifier> "=" <expr>
            | "let" <identifier> ":" <type> "=" <expr>
            | "let rec" <identifier> "=" <lambda>
            | "let rec" <identifier> ":" <type> "=" <lambda>
            | "let async" <identifier> "=" <lambda>
            | "let async" <identifier> ":" <type> "=" <lambda>

<assertion> ::= "assert" <expr> | "assert" <expr> <string>

<expr> ::=  <term> 
          | <term> "+" <expr> 
          | <term> "-" <expr>
          | <term> "==" <expr>
          | <term> "!=" <expr>
          | <term> "<" <expr>
          | <term> ">" <expr>
          | <term> "<=" <expr>
          | <term> ">=" <expr>
          | <term> "&&" <expr>
          | <term> "||" <expr>
            
<term> ::=  <factor> 
          | <factor> "*" <term> 
          | <factor> "/" <term> 
          | <factor> "%" <term> 
<factor> ::=  <literal> 
            | "(" <expr> ")" 
            | <factor> "^" <factor>
            | <factor> <userop> <factor>
            | <unaryop> <factor>
            | <factor> "(" <exprlist> ")"
            | <factor> "." <identifier>
            | <factor> "[" <expr> "]"
            | <factor> "[" <expr> ":" <expr> "]"
            | <factor> "[" <expr> ":" "]"
            | <factor> "[" ":" <expr> "]"
            | <range>
            | <if>
            | "${" <expr> "}"
            | "{" <stmtlist> "}"

<unaryop> ::= "-" | "!" | "+" | <userop>
<userop> ::= <opchar> | <opchar> <userop>
<opchar> ::=  "!" | "@" | "#" | "$" | "%" | "^" | "&" | "*" | "-" | "+" | "=" | "<" | ">" | "?" | ":" | "|" | "~"

<range> ::= "[" <expr> ".." <expr> "]"

<if> ::=  "if" <expr> "then" <expr> "else" <expr>
        | "if" <expr> "then" <expr>
        | <expr> "if" <expr> "else" <expr>

<literal> ::= <number> | <identifier> | <bool> | <list> | <lambda> | <string> | "()" | <tuple> | <record>
<string> ::= '"' <charlist> '"' | '""'
<charlist> ::= <char> | <char> <charlist>
<list> ::= "[" <exprlist> "]"
<tuple> ::= "(" <exprlist> ")"
<record> ::= "{" <recordlist> "}"
<recordlist> ::=  <identifier> "=" <expr> 
                | <identifier> "=" <expr> "," <recordlist>
                | <identifer> ":" <type> "=" <expr> 
                | <identifier> ":" <type> "=" <expr> "," <recordlist>

<lambda> ::=  "(" <typedexprlist> ")" "->" <expr>
            | "(" <typedexprlist> ")" ":" <type> "->" <expr>

<bool> ::= "true" | "false"

<number> ::= <int> | <float> | <rational>
<int> ::= <digit> | <digit> <int>
<float> ::= <int> "." <int>
<rational> ::= <int> "/" <int>
<digit> ::= "0" | "1" | "2" | "3" | "4" | "5" | "6" | "7" | "8" | "9"

<identifier> ::= <letter> | <letter> <identifier>
<letter> ::=  "a" | "b" | "c" | "d" | "e" | "f" | "g" | "h" | "i" | "j" | "k" | "l" | "m" 
            | "n" | "o" | "p" | "q" | "r" | "s" | "t" | "u" | "v" | "w" | "x" | "y" | "z"

<exprlist> ::= <expr> | <expr> "," <exprlist>
<typedexprlist> ::=  <expr> ":" <type> 
                   | <expr> ":" <type> "," <typedexprlist>
                   | <expr> "," <typedexprlist>

<type> ::=  "int" | "float" | "bool" 
          | "(" <typelist> ")" "->" <type> 
          | "[" <type> "]" | "(" <typelist> ")"
          | "{" <recordtypelist> "}"

<recordtypelist> ::= <identifier> ":" <type> | <identifier> ":" <type> "," <recordtypelist>

<typelist> ::= <type> | <type> "," <typelist>
\end{verbatim}

This BNF is represented in the F\# codebase as an AST, represented by the following type:

\begin{minted}{fsharp}
/// <summary>
/// The AST of the language.
/// </summary>
type Expr =
    | ELiteral of Literal * Type
    | EIdentifier of Token * Type option
    | EGrouping of Expr * Type option

    | EIf of Expr * Expr * Expr * Type option
    | ETernary of Expr * Expr * Expr * Type option

    | EList of Expr list * Type option
    | ETuple of Expr list * Type option

    | ECall of Expr * Expr list * Type option

    /// <summary>
    /// Indexing operation on a list or tensor.
    /// Expr (list or tensor), (index), type
    /// Allows for indexing in the form l[1]
    /// </summary>
    | EIndex of Expr * Expr * Type option

    /// <summary>
    /// Indexing with a range operation on a list or tensor.
    /// Expr (list or tensor), start, end, type
    /// Allows for indexing in the form l[..1] or l[1..2] or l[1..]
    /// </summary>
    | EIndexRange of Expr * Expr * Expr * Type option

    /// <summary>
    /// A lambda expression with a list of arguments, a body, a return type, a pure flag, and a type.
    /// </summary>
    | ELambda of (Token * Type option) list * Expr * Type option * bool * Type option * bool // bool is pure flag
    | EBlock of Stmt list * bool * Type option // bool is whether block is part of a function
    | ERange of Expr * Expr * Type option

    | ERecordSelect of Expr * Token * Type option

    /// <summary>
    /// Records represented recursively as a row type.
    /// </summary>
    | ERecordExtend of (Token * Expr * Type option) * Expr * Type option
    | ERecordRestrict of Expr * Token * Type option
    | ERecordEmpty of Type

    /// <summary>
    /// Unevaluated code block.
    /// </summary>
    | ECodeBlock of Expr

    /// <summary>
    /// A tail call (for tail recursion).
    /// </summary>
    | ETail of Expr * Type option

/// <summary>
/// A statement in the language (something that does not return a value).
/// </summary>
and Stmt =
    | SExpression of Expr * Type option
    | SVariableDeclaration of Token * Expr * Type option
    | SAssertStatement of Expr * Expr option * Type option
    | STypeDeclaration of Token * Type * Type option
    | SRecFunc of Token * (Token * Type option) list * Expr * Type option
    | SAsync of Token * (Token * Type option) list * Expr * Type option
    | SImport of Token option * string * bool * Type option // maybe binding name, module name (path), isstd, type
\end{minted}

\section{Final GUI}\label{sec:final-gui}

\section{Notable Features}\label{sec:notable-features}

\section{Lexer}\label{sec:lexer}

Initial lexer design was based on a simple regular expression based lexer, but this was later replaced with a more
functional approach using pattern matching on the input string.

The reason for this change was that the regular expression based lexer was difficult to extend and maintain due to 
the lack of type safety.
For example if we had a more general regex called before a more specific one, the more general one would always match
first, even if the more specific one should have matched.

This was solved by using a more functional approach, where the type system of F\# would inform us if a case would 
never be matched due to the order of the cases or otherwise, preventing a class of easily overlooked errors during development.

The lexer is now implemented as a recursive pattern matching function that takes a string and returns a list of 
tokens, complete with their lexeme and position in the input string.

Lexer errors are also accumulated in a list of type \textit{LexerError}, which are displayed to the user in the GUI\@.

Something of note is that the lexer parses numbers itself, rather than passing them to the parser as strings.

Additionally, due to the permittance of user defined operators, the lexer makes special considerations when lexing 
special characters, as the distinction between a built-in operator (with precedence) and a user defined operator (
currently without taking precedence into account) is made during lexing.

Furthermore, both block comments (\textit{/* */}) and line comments (\textit{//}) are handled by the lexer by ignoring
the contents of the comment.
In future, it may be interesting represent comments as a token in the AST, allowing for systems such as documentation
generation or automatic formatting to be implemented.

\section{Parser}\label{sec:parser}

The parser is implemented using Pratt parsing\citep{pratt1973top}, which is a top-down operator precedence parsing 
method that allows for easy extension and modification of the grammar.

It works by assigning a precedence to each token, as well as functions specifying how to parse the token when 
encountering it in a prefix, infix or postfix position.

For example, take the expression $2 + 3 * 4$.

The parser would first encounter the number \textit{2}, which has a precedence of 0 and a prefix function that
simply returns the number.

Thus, the current state of the parser is $2$.

The parser would then encounter the operator \textit{+}, which has a precedence of 1 and a left associative infix
function that takes the left hand side and the right hand side and returns a binary expression node.

The parser then attempts to parse the right hand side of the operator with a precendence level higher than the
plus operator, as Pratt parsing must ensure that higher precedence operations (such as multiplication) are parsed
first.

The parser would then encounter the number \textit{3}, which again is treated as a literal and returned.

The parser then encounters the operator \textit{*}, which has a precedence of 2 (higher than the plus operator) and 
as such the parser cannot yet resolve the \textit{+} operator; it must handle the higher precedence multiplication
operator first.

The parser saves the left hand side (the number 3) and then parses the right hand side of the multiplication 
operator using a precedence level higher than the multiplication operator.

It encounters the number \textit{4}, which is returned as a literal.

The parser then returns the binary expression node for the multiplication operator, with the left hand side being
the number 3 and the right hand side being the number 4.

The parser then returns to the plus operator, which can now be resolved as the left hand side is the number 2 and the
right hand side is the result of the multiplication operator.

This is a simple example, but Pratt parsing can handle more complex expressions with ease, such as nested
expressions and function calls.

Using Pratt parsing has improved the extensibility of the parser, as adding new operators or changing the grammar
is as simple as adding a new case to the parser.

A slight limitation is during ambiguity, such as the \textit{(} symbol, which can be used for a grouping, a lambda 
definition, a tuple or a \textit{unit} type when encountered in the prefix position.
This is resolved through a state machine approach, where the parser can move around the state at will, allowing 
lookahead and backtracking in order to reach a point where the ambiguity is resolved.

In order to simplify the Virtual Machine\ref{sec:virtual-machine}, the parser parses all binary and unary operations 
as function calls, with the operator as the function name.

In order to make type inference simpler for operators that are overloaded for both unary and binary operations (such 
as the \textit{-} operator), the operator itself keeps track of the manner in which it is called (unary or binary) and
returns the appropriate AST node. 
This allows for easier type inference (as the names of the overloaded functions are different), 
and simplifies the bytecode generation process by removing ambiguity in the AST\@.

This idea could possibly be extended to allow other overloaded function names (with varying numbers of arguments or 
arguments of different types).

\section{AST}\label{sec:expression}

The AST of the language is represented as a list of statements, where a statement is either expression, a
variable assignment or an other statement type.
It is typed (after type inference\ref{sec:type-inference}) in order to allow for easier optimisation and
bytecode generation.

The AST representation is given in section~\ref{sec:final-bnf}.

\section{Type Inference}\label{sec:type-inference}
Vec3 is a statically typed language, with full type inference.
The type inference algorithm is based on Hindley-Milner type inference\citep{sulzmann2000general}, with some
modification to support the non-ML style syntax, and extended to support \textit{row polymorphism}\citep{morris2019abstracting} (\ref{subsec:row-polymorphism}), \textit{gradual typing}\citep{garcia2016abstracting}
(\ref{subsec:gradual-typing}), \textit{recursive bindings} (\ref{subsec:recursive-bindings}), 
\textit{vector length encoding} (\ref{subsec:vector-length-enc}) and a seemingly unique method of supporting ad-hoc 
polymorphism named \textit{constraints} (\ref{subsec:constraints}).

The reason for implementing strong type inference due to the \textit{Semantic Soundness Theory}\citep{timany2024logical}, which states that a \textit{well-typed program cannot go wrong}.

Of course this is not strictly true in practice due to external factors, but it is certainly true that strong typing
rules out a large class of errors, most of which human, and as such it is a valuable tool for a maths language to
have as the user is less likely to make trivial mistakes.

Another thing to note is that in order to make the language more intuitive to use, the \textbf{integer} type will 
coerce into any other number type (\textbf{float, rational, complex}). 
This allows for expressions such as $5.0 ^ 5$ type checking successfully, with the result being a float.

\subsection{Type Inference Algorithm}\label{subsec:type-inference-algorithm}

The algorithm used to infer types is based on Algorithm W\citep{milner1978theory}.
The general idea is to assign the widest type possible for a given node in the AST, which is generally a
\textit{type variable}, which is a type used to represent a type that can be unified with any other type (a generic
type).
The node's children are then inferred, and the types of the children are unified with the parent node.
If the types cannot be unified, then the program is ill-typed.
Unification is the process of finding the most general type that can be assigned to two types, and is a key part of
the algorithm.

For example, unifying \textit{int} and \textit{int} would result in \textit{int}, as this is the most general type that
can be assigned to both.

Contrasting this, unifying \text{int} and a type variable \textit{a}, would result in \textit{int}, and then the type
variable \textit{a} would have to be substituted with \textit{int} throughout the program (because \textit{int} is the
most general type that can be assigned to \textit{a}).
It works bottom-up as only a few types are known at the start, such as the types of literals and the types of
built-in functions.

\paragraph{Algorithm Implementation}\label{par:algorithm-implementation}
A simplified version of the algorithm, with some details omitted for brevity, is shown in Algorithm~\ref{alg:algorithm}.

\begin{algorithm}
    \caption{Type Inference Algorithm}
    \begin{algorithmic}[1]
        \Function{$unify$}{$type1,\ type2$}
            \If{$type1\ \textbf{is}\ type\ variable$}
                \State $type1 \gets type2$
            \EndIf
            \If{$type2\ \textbf{is}\ type\ variable$}
                \State $type2 \gets type1$
            \EndIf
            \If{$type1\ \textbf{is}\ function\ type\ and\ type2\ \textbf{is}\ function\ type$}
                \State $unify\ paramTypes$
                \State $unify\ returnTypes$
            \EndIf
            \If{$type1$\ \textbf{is}\ not\ equal\ to\ $type2$}
                \State \textbf{error}
            \EndIf
        \EndFunction

        \Function{$infer$}{$expr,\ env$}
            \If{$expr\ \textbf{is}\ literal$}
                \State \Return $type\ of\ literal$
            \EndIf
            \If{$expr\ \textbf{is}\ variable$}
                \State $T \gets lookup\ variable\ env$
                \State \Return $type$
            \EndIf
            \If{$expr\ \textbf{is}\ function\ call$}
                \State $funcType \gets infer\ function$
                \State $argTypes \gets infer\ arguments$
                \State $funcType \gets unify\ paramTypes\ argTypes$
                \State $returnType \gets return\ type\ of\ funcType$
                \State \Return $returnType$
            \EndIf
            \If{$expr\ \textbf{is}\ binding$}
                \State $bodyType \gets infer\ body$
                \State $env \gets add\ binding\ bodyType\ environment$
                \State \Return $bodyType$
            \EndIf
            \If{$expr\ \textbf{is}\ lambda$}
                \State $argTypes \gets new\ type\ variables$
                \State $bodyEnv \gets add\ arguments\ to\ environment$
                \State $bodyType \gets infer\ body\ with\ bodyEnv$
                \State $funcType \gets argTypes+bodyType$
                \State \Return $funcType$
            \EndIf
        \EndFunction
    \end{algorithmic}\label{alg:algorithm}
\end{algorithm}

As shown, it is an incredibly simple yet powerful algorithm, and is the basis for many modern type inference
algorithms, such as that of F\# and OCaml.

\paragraph{Bindings}\label{par:bindings}
Generally in implementations of \textit{Algorithm W}, after type inference for a given binding has taken place a
process known as \textit{generalisation} occurs.
This is the process of replacing all type variables in the type of
the binding with \textit{forall} quantifiers, which is a way of saying that the type is polymorphic and can therefore be
instantiated with any type.

However, this was not necessary in our implementation as we don't specialise bindings during the instantiation of
types (such as during calls), we simply infer the type of the call and check it against the type of the binding, so
generalisation is not necessary.

\subsection{Gradual Typing}\label{subsec:gradual-typing}

Gradual typing is a type system that allows for the gradual transition from dynamic typing to static typing.
This is useful in a language like Vec3 as it allows for the user to write code without having to worry about types
allowing for quick prototyping, but then add types later to ensure correctness.
Users have the option of adding types to their code in the form $let\ x: int = 5$, and the type inference algorithm
will check that the type of the expression matches the type given.

The type $any$ can also be used, which represents a dynamic type that can be unified with any other type.
This disables the safety guarantees of the type system, but can be useful as mentioned above for quick prototyping.
One thing to note however is that the $any$ type is infectious, meaning that if a type is inferred to be $any$ then
the type of the parent node will also be $any$.

\subsection{Row Polymorphism}\label{subsec:row-polymorphism}

Row polymorphism is a form of polymorphism that allows for the definition of functions that operate on records with
a certain set of fields, but can also operate on records with additional fields.
It can be considered both a form of structural typing like that of TypeScript\citep{bierman2014understanding}, and a
form of subtyping.

For example, consider the following function:

\begin{minted}{fsharp}
let f = (x) -> x.a
\end{minted}

This function takes a record with a field \textit{a} and returns the value of that field.

Now consider the following record:

\begin{minted}{fsharp}
let r = {a = 5, b: int = 6}
\end{minted}

The function \textit{f} can be called with \textit{r} as an argument as \textit{r} has a field \textit{a}, and the
function will return 5.
This is a powerful feature as it allows for the definition of functions that operate on a wide range of records.
The reference algorithm given by \citet{morris2019abstracting} was used as a basis for the implementation of row
polymorphism in Vec3, however without record restriction as it was not necessary for the language.

The algorithm works by assigning a \textit{row variable} to each record type, which is a type variable that represents
the fields of the record, where a record is represented in the type system as an extension of another record, or the empty record.
The algorithm then unifies the row variables of the record types, and if the unification is successful then the
records are considered to be the same type.

A reference implementation of the algorithm is given in Algorithm~\ref{alg:row-polymorphism}.

\begin{algorithm}
    \caption{Row Polymorphism Algorithm}
    \begin{algorithmic}
        \Function{$unify$}{$type1,\ type2$}
            \If{$type1\ \textbf{is}\ row\ variable$}
                \State $type1 \gets type2$
            \EndIf
            \If{$type2\ \textbf{is}\ row\ variable$}
                \State $type2 \gets type1$
            \EndIf
            \If{$type1\ \textbf{is}\ record\ type\ and\ type2\ \textbf{is}\ record\ type$}
                \State $unify\ row\ variables$
            \EndIf
            \If{$type1$\ \textbf{is}\ not\ equal\ to\ $type2$}
                \State \textbf{error}
            \EndIf
        \EndFunction
    \end{algorithmic}
    \label{alg:row-polymorphism}
\end{algorithm}

With some creativity, row polymorphism can be used to represent semi-algebraic data types or tagged unions.
For example, consider the built-in \textit{on} function (used to add event listeners for shapes)\ref{sec:drawing}:

\begin{minted}{fsharp}
on(shape, Keys.Down, (state) -> ...)
\end{minted}

The function expects a shape reference, a key, and a function that takes a state.

The implementation of the keys record is hidden from the user, but could well be implemented as a record with a field
for each key, where each key is a record that contains a field 
\textit{43hr4h54j3} (a unique identifer for the keys record) and the \textit{on} function could have a type of:
\begin{minted}{fsharp}
let on = (shape, key: { 43hr4h54j3: int }, func) -> ...
\end{minted}

This has pretty good type safety, as the function will only accept keys with said field, which is hidden from the user.
This doesn't have quite as good safety guarantees as a true algebraic data type, i.e.\ in the form 
$data\ bool\ =\ True\ |\ False$, but is certainly safer than say C enums, preventing mistakes such as using incorrect 
argument order.

\subsection{Recursive Bindings}\label{subsec:recursive-bindings}

Due to the fact that everything is immutable in Vec3, the simplest way to ensure Turing completeness is to allow for 
recursive bindings (i.e.\ functions that call themselves).
This is a powerful feature as it allows for the definition of functions that operate on recursive data structures, such
as trees and lists.

The type inference algorithm was modified to support recursive bindings, as the standard algorithm would not be able to
infer the type of a recursive due to the fact that the binding would not be in the environment when the type of the
function was inferred (all functions are lambdas, and therefore are not assigned to a binding until after declaration).
Hence, recursive functions were introduced as a separate statement in the grammar of the language, and the type
inference algorithm was modified to support them by adding the binding to the environment before inferring the type of
the function.

\subsection{Constraints}\label{subsec:constraints}

Due to the restrictiveness of the standard Hindley-Milner type inference algorithm, it is not possible to support ad-hoc
polymorphism (i.e.\ overloading) without some modification.
For example, OCaml\citep{ocamlDocs} does not support ad-hoc polymorphism and instead uses, for example, the \textit{+}
operator for integer addition and the \textit{+.} operator for float addition, which is not ideal for this language 
as it would be unintuitive for a mathematician.

Examples of ML style languages that do support ad-hoc polymorphism are F\#, which uses static member functions on 
types to achieve operator overloading\citep{fsharpdocs}, and Haskell, which uses type classes\citep{haskellDocs} (constructs that define behaviour for a type, similarly to interfaces in object-oriented languages).
The way this issue was solved in Vec3 is by introducing the concept of \textit{Constraint types}, which could be 
likened to a slightly less powerful version of type classes in Haskell.
A constraint is a type that is defined by a type variable, and a function of type \textit{$Type \rightarrow bool$}.
During unification, if a type is unified with a constraint type, then the function is called with the type, and if it
returns true then the unification is successful and the type variable that the constraint holds is unified with the
type.

For example, consider the type of the \textit{+} operator:

\begin{minted}{fsharp}
(+) :: Constraint (a, supportsArithmetic) -> Constraint (a, supportsArithmetic) -> a
\end{minted}

Then, when the operator is used with say two ints, the first constrain would be unified with the type 
\textit{int} (as the int type passes the \textit{isArithmetic} function), and the type variable \textit{a} would be 
replaced with \textit{int}.
The second int would then be unified with the type \textit{int}, and the unification would be successful.
However, if the operator was used with a rational and a float, the first constraint would be unified with the type
\textit{rational}, replacing the type variable \textit{a} with \textit{rational}, and the second constraint would be 
unified with the type rational, which does not unify with float, and so the unification would fail.

This type constrain acts as a normal type, allowing for user defined ah-hoc functions, such as:

\begin{minted}{fsharp}
let double = (x) -> x + x
\end{minted}

This function can be called with any type that supports arithmetic, and the type inference algorithm will infer the
type of the function as \textit{$Constraint\ (a,\ supportsArithmetic)\ \rightarrow\ a$}.

Something else unique is the concept of a \textit{transformation}, which is a function of the constraint type that
transforms the type into another type.
This was necessary due to functions such as \textit{append}, which appends two lists.
Due to the fact that the length of the list is encoded in the type, without a transformation the arguments could 
only unify if the lists were of the same length, which is an unnecessary restriction.
As such, the transformation function is used to transform the dimensions of the first type into a list without 
dimension restrictions so unity can occur.

A current limitation of this system is that the user cannot define their own constraints, and the only constraints 
present are those built into the language (such as operators).
This is a feature that could be added in the future, but was not necessary for the current implementation of the
language.

\subsection{Vector Length Encoding}\label{subsec:vector-length-enc}

Another key feature present in the type system is the encoding of vector lengths.

The type of a vector looks like \textit{Vector of Type * Dims}, where \textit{Type} is the type of the elements, and \textit{Dims} is an integer representing the number of dimensions.
This means that a vector \textit{[1,2,3]} is inferred to be of type \textit{Vector of int * 3}.
This allows for the type system to catch errors such as adding two vectors of different lengths, or only allowing the \textit{cross product} 
function to be called on vectors of length 3.

This is a powerful feature as it allows for the type system to catch errors that would otherwise only be caught at 
runtime with standard type inference.
In its current state, it also allows for slight \textit{refinement types}\citep{freeman1991refinement}, which are types that are dependent on values, such as the length of a vector.

Examples of this catching an otherwise runtime error is shown in the following code:

\begin{minted}{fsharp}
let a = [1,2,3]
let b = [1,2]
let c = a + b // Error: Vectors must be of the same length
    
let d = [1,2,3]
let e = d[4] // Error: Index out of bounds
\end{minted}

The latter example is currently very simple, and only catches out of bound errors during indexing with constant
values, but could be extended to catch more complex errors in the future.

Another use case for this is during matrix operations, where the type system can ensure that the dimensions of the
matrices are correct, preventing errors such as finding the transpose of a non-square matrix:

\begin{minted}{fsharp}
let a = [[1,2,3], [4,5,6]]
let b = transpose(a) // Error: Matrix must be square
\end{minted}

Furthermore, the inner tensors of a matrix are also encoded with their dimensions, allowing for the type system to catch
errors such as accidentally creating a matrix with rows of different lengths:

\begin{minted}{fsharp}
let a = [[1,2,3], [4,5]] // Error: Rows must be of the same length
\end{minted}

One thing to note however is that the dimensions of a vector are lost fairly easily, for example during 
\textit{cons}, as it is not powerful enough to infer the length of the resulting vector.

Having full dependent types would solve this issue, but would be overkill for this language, and would make the
type system much more complex (likely requiring a theorem prover and types as values).

\subsection{Function Purity}\label{subsec:function-purity}

The purity of a function is also determined during type inference, with the type of a function being inferred as
pure if it is made up of only pure functions, with the base pure functions being built in.
This allows for, for example, the \textit{plotFunc} (\ref{sec:plotting}) ensuring that only pure functions can be 
passed to it, preventing a user from passing a function that has side effects (which would likely cause a runtime 
error otherwise).
It also allows for easy dead code elimination, as a call to a function that has no side effects can be removed if the
result is not used.


\section{Optimisation}\label{sec:optimisation}

Before compilation, the AST is optimised by removing dead code and constant folding.

\subsection{Dead code elimination}\label{subsec:dead-code-elimination}

Dead code elimination is performed on the AST by removing any statements that are not used.
For example, if a variable is declared but never used, the variable declaration is removed or if an expression is
written but never used, the expression is removed.

This is accomplished by through static analysis of the AST, where the following process is repeated until no more dead
code can be removed:

\begin{algorithmic}
    \While{Dead code can be removed}
        \For{Each node in the AST}
            \If{Node is a statement}
                \If{Statement is not used}
                    \State Remove statement
                \EndIf
            \ElsIf{Node is an expression}
                \If{Expression is not used}
                    \State Remove expression
                \EndIf
            \ElsIf{Node is a binding}
                \If{Variable is not used}
                    \State Remove binding
                \EndIf
            \EndIf
        \EndFor
    \EndWhile
\end{algorithmic}

The process is repeated until no more dead code can be removed, allowing for long chains of dead code to be 
removed (for example if a variable is used in a function that is never called, the function would first be removed
and then the variable).
It is to be noted that variable assignments are never removed during DCE when running the code editor due to the 
attached REPL, as the user may wish to use the variable in the REPL\@, or when running code blocks in the notebook 
view\ref{sec:notebook-view} as the variable may be used in a later code block.
However, DCE can be aggressively performed when transpiling to C\ref{sec:transpiler}, as the user is not expected to 
interact with the generated C code.

\subsection{Constant folding}\label{subsec:constant-folding}

Constant folding is performed on the AST by evaluating constant expressions at compile time, such as $2 + 2 \ra 4$.
This is accomplished using the initial interpreter implementation, which recursively evaluates the AST and replaces
constant expressions with their evaluated value.
Only constants are evaluated, and thus no variable resolution is performed due to the cost of this operation.

\section{Initial Design of the Bytecode Virtual Machine and Compiler}\label{sec:initial-design-of-the-bytecode-virtual-machine-and-compiler}

The core of the Vec3 interpreter was a transition from a tree-walk interpreter to a more efficient stack-based virtual machine. 
This involved two primary components: a compiler to translate Vec3 source code into bytecode, and a virtual machine to execute that bytecode. 
The fundamental goal was to achieve faster execution speeds by working with a compact and streamlined instruction set.

The bytecode itself was a sequence of instructions, each represented by an operation code (opcode) and, potentially, operands that the opcode would act upon. 
These opcodes, defined in the \texttt{OP\_CODE} type, covered a range of operations necessary for a fully functional language. 
This included instructions for pushing constants onto the stack (\texttt{CONSTANT}, \texttt{CONSTANT\_LONG}), performing arithmetic (\texttt{ADD}, \texttt{SUBTRACT}, \texttt{MULTIPLY}, \texttt{DIVIDE}, \texttt{NEGATE}), managing control flow (\texttt{RETURN}, \texttt{JUMP}, \texttt{JUMP\_IF\_FALSE}, \texttt{LOOP}), handling boolean logic (\texttt{NIL}, \texttt{TRUE}, \texttt{FALSE}, \texttt{NOT}, \texttt{EQUAL}, \texttt{GREATER}, \texttt{LESS}), manipulating the stack (\texttt{POP}), working with global and local variables (\texttt{DEFINE\_GLOBAL}, \texttt{GET\_GLOBAL}, \texttt{SET\_GLOBAL}, \texttt{GET\_LOCAL}, \texttt{SET\_LOCAL}), outputting values (\texttt{PRINT}), and calling functions (\texttt{CALL}). Two functions, \texttt{opCodeToByte} and \texttt{byteToOpCode}, handled the conversion between these symbolic opcodes and their corresponding byte representations, ensuring a compact bytecode format.

Compiled code, along with associated data, was organized into \texttt{Chunk} structures. 
Each chunk contained a \texttt{Code} array, holding the bytecode instructions as a sequence of bytes. 
A \texttt{ConstantPool} array stored constant values referenced by the instructions, enabling efficient reuse of values like numbers and strings. 
To aid in debugging, a \texttt{Lines} array mapped bytecode offsets to their corresponding line numbers in the original source code. 
Functions like \texttt{emptyChunk}, \texttt{writeChunk}, \texttt{addConstant}, \texttt{writeConstant}, and \texttt{getLineNumber} provided an interface for creating and manipulating chunks.

The compiler's role was to transform the abstract syntax tree (AST) representation of Vec3 code into this bytecode format. 
It maintained a \texttt{CompilerState} to track the chunk being generated, local variables within the current scope, the current scope's nesting depth, and the line number being processed. 
The compilation process involved a recursive descent through the AST. Functions like \texttt{compileStmt} and \texttt{compileExpr} recursively processed statements and expressions, respectively. 
For each AST node encountered, the compiler emitted corresponding bytecode instructions using helper functions like \texttt{emitByte}, \texttt{emitBytes}, \texttt{emitConstant}, and \texttt{emitOpCode}.

Variable declarations were handled by \texttt{compileVariableDeclaration}, which added the variable to the \texttt{Locals} map in the \texttt{CompilerState}. 
This map stored local variables and their corresponding slot indices on the virtual machine's stack. 
Control flow instructions, like jumps and loops, were initially emitted with placeholder offsets. 
These placeholders were later patched with the correct offsets once the target locations were determined.

\textbf{Error Handling}

Error handling during compilation was managed using the \texttt{CompilerResult} type. 
This allowed the compiler to either return a successful result along with an updated \texttt{CompilerState} or an error along with a descriptive message and the state at the point of the error.

\textbf{Debugging}

To facilitate debugging, a disassembler was implemented. 
Functions like \texttt{disassembleInstruction} and \texttt{disassembleChunk} took the compiled bytecode and produced a human-readable representation, showing the instructions, their operands, and their associated source code line numbers. 
This was invaluable for understanding the generated bytecode and identifying potential issues.

\section{Virtual Machine}\label{sec:virtual-machine2}

The Virtual Machine (VM) was responsible for executing the compiled bytecode. 
It was designed as a stack-based machine, meaning that it used a stack to store intermediate values during computation. 
The VM's state was represented by the \texttt{VM} type, defined as follows:

\begin{minted}{fsharp}
type VM = {
    Chunk: Chunk
    IP: int
    Stack: ResizeArray<Value>
    ScopeDepth: int  
}
\end{minted}

\noindent where:

*   \texttt{Chunk} held the bytecode and associated data (constant pool, line information) currently being executed.
*   \texttt{IP} (Instruction Pointer) was an integer representing the index of the next bytecode instruction to be executed.
*   \texttt{Stack} was a dynamically sized array used to store values during computation. Operations like arithmetic, comparisons, and function calls would push and pop values from this stack.
*   \texttt{ScopeDepth} tracked the current level of scope nesting.

The \texttt{createVM} function initialized a new VM instance with a given chunk. The core of the VM was the \texttt{run} function, a recursive loop that fetched, decoded, and executed instructions until a \texttt{RETURN} instruction was encountered or an error occurred.

Key helper functions included:

*   \texttt{push}: Pushed a value onto the stack.
*   \texttt{pop}: Popped a value from the stack.
*   \texttt{peek}: Looked at a value on the stack at a given position without removing it.
*   \texttt{readByte}: Read the byte at the current instruction pointer and incremented the IP.
*   \texttt{readConstant}: Read a constant index from the bytecode, fetched the corresponding value from the constant pool, and pushed it onto the stack.
*   \texttt{readConstantLong}: Similar to \texttt{readConstant}, but for constants that required a larger index.
*   \texttt{binaryOp}: Performed a binary operation on the top two values on the stack.

The \texttt{run} function used a match expression to dispatch to the appropriate code based on the current opcode. Each opcode case handled a specific instruction, potentially manipulating the stack, performing calculations, or managing control flow.

The \texttt{interpret} function provided the main entry point for executing a chunk of bytecode. It first disassembled the chunk for debugging purposes, then created a new VM instance and called \texttt{run} to start the execution process.

In conclusion, this initial design established a solid foundation for a stack-based virtual machine and its associated compiler. It emphasized a clean separation of concerns between compiling and executing code, a compact bytecode representation, and a focus on essential features for a functional language. The inclusion of debugging tools, a well-defined error-handling mechanism, and a dedicated virtual machine for execution further contributed to the robustness and efficiency of the system.

\section{Virtual Machine}\label{sec:virtual-machine}

\section{Prelude}\label{sec:prelude}

A prelude is implicitly included in every program, which contains some useful functions defined in the language, as 
well as wrappers for the built-in functions of the Virtual Machine. Initially, defining the following instructions:

Notable functions include:

\begin{itemize}
    \item \textit{map}, \textit{fold} and \textit{filter} functions for lists.
    \item \textit{range} function for generating a list of numbers.
    \item \textit{sqrt}, \textit{cubeRoot} which are specialisations of the \textit{root} function.
    \item \textit{head}, \textit{tail} and \textit{len} functions for lists.
    \item \textit{findIntegral} function for finding the integral of a function.
\end{itemize}

We felt it was useful implementing these in-language functions as it allows for more concise and readable code, as
well as showcasing the power of the language.

\section{Plotting}\label{sec:plotting}

The plotting system is implemented using ScottPlot\citep{scottPlot}, a plotting library for .NET\@.

The functionality is exposed to the user through 3 built-in functions: \textit{plot}, \textit{plotFunc} and 
\textit{plotFuncs}.

\textit{plot} takes in a record of configuration options of the following type:

\begin{minted}{fsharp}
    type PlotConfig = {
        title: string,
        XValues: [float],
        YValues: [float],
        ptype: "bar" | "scatter" | "signal",
    }
\end{minted}

The resulting plot is then displayed in a separate window based on these configuration options.

\todo{Add images of plots}

The \textit{plotFunc} function takes in a string title and a pure function of type \textit{float -> float}.
The function is then plotted on the graph with an infinite range of x values.
Optionally, the user can also specify two more float values, \textit{start} and \textit{end}, in which case the 
integral of the function is calculated and displayed on the graph.

\todo{Add images of plots}

The \textit{plotFuncs} function takes in a string title and a list of pure functions of type \textit{float -> float}.
This allows for multiple plots to be placed on the same window, which we felt was valuable for comparing functions 
or plotting derivatives.

The plot windows also have an input at the bottom, which allows for the user to input a function and have it plotted
on command.
This is useful for quick visualisation of functions, and allows for a more interactive experience.

\section{Drawing}\label{sec:drawing}

As well as plotting, the user also has the option of drawing arbitrary shapes on a canvas, and attaching event 
listeners to them.

This is done by means of the \textit{draw} function, which takes in a record of configuration options of the following
type:

\begin{minted}{fsharp}
    type DrawConfig = {
        x: float,
        y: float,
        width: float,
        height: float,
        color: string,
        shape: "rectangle" | "circle",
        trace?: bool, 
    }
\end{minted}

Or a list of the above record type, allowing for multiple shapes to be drawn on the same canvas.

The \textit{draw} function then returns a unique identifier for the shape, which can be used to attach event 
listeners, allowing for movement of the shape through key presses.

The following example attached event listeners to a shape which moves it left and right following the \textit{cos} curve:

\begin{minted}{fsharp}
on(id, Keys.Right, (state) -> { x = state.x + 10.0, y = cos(state.x) * 10.0 + 100.0 })
on(id, Keys.Left, (state) -> { x = state.x - 10.0, y = cos(state.x) * 10.0 + 100.0 })
\end{minted}

Where keys is a record defined in the prelude of the language (see~\autoref{sec:prelude}).

Additionally, the \textit{trace} option allows for the shape to leave a trail behind it, which can be useful for
animations or visualising movement.

\todo{Add images of drawings}

\section{Notebook View}\label{sec:notebook-view}

The notebook view is a feature that allows the user to write code in a more interactive way, similar to Jupyter
notebooks\citep{Jupyter}.

\section{Transpiler}\label{sec:transpiler}

The user also has the option of transpiling their code to C, which can then be compiled and run as a standalone
executable, allowing for faster execution of the code which is important for larger or more computationally
intensive programs.

\section{Code architecture}\label{sec:code-architecture}

\chapter{Discussion, conclusion and future work}\label{ch:discussion-conclusion-and-future-work}

\section{Discussion and Conclusion}\label{sec:discussion}

Everything that we set out to do has been completed to a high standard, and more.
The language is Turing complete, has a GUI, plotting, static type inference, first class functions, compound data types,
recursion, error handling, a transpiler to C, and a standard library.
It also has importing, control flow, async functions, and a large number of built-in functions, as well as support 
for rational and complex numbers.

We feel that the project has been a success, and we are proud of what we have achieved.

\section{Future Work}\label{sec:future-work}

An infinite number of more maths functions could be added to the language, such as more matrix operations, more
trigonometric functions, and more complex number operations, such as plotting of complex numbers.
Additionally, multi-parameter function plotting would be a useful feature, allowing for plotting of custom circles and
ellipses, for example.

However, the language is to a standard wherein lots of these functions can be implemented in the language itself, by 
the user for example, so adding more built-in functions is something that can be done externally and incrementally.

A more IDE like experience, comparable to another maths language like R or MATLAB would be useful, as it would allow 
users a more complete experience, outside of prototyping.
Obviously we have imports, both standard library and user made files, but no way to save progress or set a working 
directory, so this would be a useful feature to add.


\bibliographystyle{apalike}

\bibliography{References}

\appendix
\chapter{Contributions}\label{ch:contributions}

50/50

\section{Jacob Edwards}\label{sec:jacob-edwards}

\begin{itemize}
    \item Lexer
    \item Parser
    \item Compiler
    \item Type inference
    \item Testing
    \item Documentation
\end{itemize}

\section{Jamie Wales}\label{sec:jamie-wales}

\begin{itemize}
    \item GUI
    \item Plotting
    \item Transpiler
    \item Testing
    \item Documentation
    \item Parser
    \item Compiler
    \item Lexer
\end{itemize}

\chapter{Testing}\label{ch:test}

The language has been tested thoroughly using .Net's built in unit testing framework, NUnit.
Unit tests have been written for the lexer, parser, compiler, type inference, compiler and transpiler, and are 
within the Test project.

Additionally, tests have been written in the language itself to test the language's features.

\section{Arithmetic Expression Testing}\label{sec:arithmetic-expression-testing}

Tests can be found in table~\ref{tab:table2}.

\begin{table}[h]
    \caption{Arithmetic expression tests. Note that floating pointing values are accurate to three decimal places for the fractional part. ResE is expected result and ResA is actual result. \\}
    \begin{tabular}{|p{1.8in}|p{0.5in}|p{0.4in}|p{0.6in}|p{1.4in}|} \hline
    Expression & ResE & ResA& Pass/Fail & Action/comment \\ \hline \hline
    $5*3+(2*3-2)/2+6$ & 23 & 23 & Pass & BIDMAS \\ \hline
    $9-3-2$ & 4 & 4 & Pass & left assoc.\  \\ \hline
    $10 / 3$ & 3 & 3 & Pass & int division  \\ \hline
    $10 / 3.0$ & 3.333 & 3.333 & Pass & float division \\ \hline
    $10\%3$ & 1 & 1 & Pass & Modulus \\ \hline
    $10 - -2$ & 12 & 12 & Pass & unary minus\\ \hline
    $-2 + 10$ & 8 & * & Pass & Unary minus \\ \hline
    $3*5\verb|^|(-1+3)-2\verb|^|2*-3$ & 87 & 87 & Pass & power test \\ \hline
    $-3\verb|^|2$ & -9(*) or 9 & 9 & Pass & precedence \\ \hline
    $-7\%3$ & 2(*) or -1 & -1 & Pass & precedence (*)Python\\ \hline
    $2*3^2$ & 18 & 18 & Pass & precedence pow > mult \\ \hline
    $3*5\verb|^|(-1+3)-2\verb|^|-2*-3$ & 75.750 or 75 & 75 & Pass & Complex expression \\ \hline
    $3*5\verb|^|(-1+3)-2.0\verb|^|-2*-3$ & 75.750 & -75.750 & & \\ \hline
    $(((3*2--2)))$ & 8 & 8 & Pass & \\ \hline
    $(((3*2--2))$ & Error & Error: Expected \("\)argument or \('\)(' after call\("\) & & syntax error \\ \hline
    $-((3*5-2*3))$ & -9 & -9 & Pass & minus expression \\ \hline
        % rational numbers
    $1/2$ & 1/2 & 1/2 & Pass & Rational number \\ \hline
    $1/2 + 1/2$ & 1/1 & 1/1 & Pass & Rational number \\ \hline
    $1/2 + 1/3$ & 5/6 & 5/6 & Pass & Rational number \\ \hline
    $1/2 * 4/7$ & 2/7 & 2/7 & Pass & Rational number \\ \hline 
    $4/5 * 2/3 + 1/2 / 3/4$ & 6/5 & 6/5 & Pass & Rational number expressions \\ \hline
    $1/2 * 4 + 3 - 1/3$ & 14/3 & 14/3 & Pass & Rational number with integer \\ \hline
        % complex numbers
    $1+2i$ & 1+2i & 1+2i & Pass & Complex number \\ \hline
    $1+2i + 3+4i$ & 4+6i & 4+6i & Pass & Complex number \\ \hline
    $1+2i * 3+4i$ & 1+10i & 1+10i & Pass & Complex number \\ \hline
    \end{tabular}
    \label{tab:table2}
\end{table}

Other tests have been written in the language itself.
An example is shown below:

\begin{minted}{fsharp}
// arithmetic tests
assert 1 * 1 == 1, "Multiplication failed"
assert 1 / 1 == 1, "Division failed"
assert 1 % 1 == 0, "Modulus failed"
assert 1 ^ 1 == 1, "Exponentiation failed"
assert 1 == 1, "Equality failed"
\end{minted}

\section{Lexer testing}\label{sec:lexer-testing}

Tests for the lexer are written using NUnit and test the lexer's ability to correctly tokenise the input string.
They can be found in the source code or in the documentation.

\section{Parser testing}\label{sec:parser-testing}

Tests for the parser are written using NUnit and test the parser's ability to correctly parse the input string.
They can be found in the source code or in the documentation.

\section{Variable assignment testing}\label{sec:variable-assignment-testing}

Tests can be found in table~\ref{tab:variables}.

\begin{table}[h]
    \caption{Variable expression tests. Note that floating pointing values are accurate to three decimal places for the fractional part. ResE is expected result and ResA is actual result. \\}
    \begin{tabular}{|p{1.8in}|p{0.5in}|p{0.4in}|p{0.6in}|p{1.4in}|}
        \hline
        Expression & ResE & ResA & Pass/Fail & Action/comment \\
        \hline \hline
        $let\ x = 3$ & & & & \\
        $(2*x)-x\verb|^|2*5$ & -39 & -39 & Pass & var assign \\
        \hline
        $let\ x = 3$ & & & & \\
        $(2*x)-x\verb|^|2*5 / 2$ & -16 & & & \\
        \hline
        $let\ x = 3$ & & & & \\
        $(2*x)-x\verb|^|2*(5 / 2)$ & -12 & -12 & Pass & \\
        \hline
        $let\ x = 3$ & & & & \\
        $(2*x)-x\verb|^|2*5 / 2.0$ & -16.5 & -16.5 & Pass & Conversion \\
        \hline
        $let\ x = 3$ & & & & \\
        $(2*x)-x\verb|^|2*5\%2$ & 5 & 5 & Pass & \\
        \hline
        $let\ x = 3$ & & & & \\
        $(2*x)-x\verb|^|2*(5\%2)$ & -3 & -3 & Pass & \\
        \hline
    \end{tabular}
    \label{tab:variables}
\end{table}

More variable testing are written in the language itself.
An example is shown below:

\begin{minted}{fsharp}
// variable tests
let x = 1
assert x == 1, "Assignment failed"
assert x^2 == 1, "Assignment with expression failed"
assert x^2 + 1 == 2, "Assignment with expression failed"
\end{minted}

\section{Function testing}\label{sec:function-testing}

Tests can be found in table~\ref{tab:functions}.

\begin{table}[h]
    \caption{Function tests. Note that floating pointing values are accurate to three decimal places for the fractional part. ResE is expected result and ResA is actual result. \\}
    \begin{tabular}{|p{1.8in}|p{0.5in}|p{0.4in}|p{0.6in}|p{1.4in}|} \hline
    Expression & ResE & ResA& Pass/Fail & Action/comment \\ \hline \hline
    $let\ f = (x)\ \rightarrow x^2$ & & & & \\ 
    $f(3)$ & 9 & 9 & Pass & function \\ \hline
    $let\ f = (x)\ \rightarrow x^2$ & & & & \\ 
    $f(3) + f(2)$ & 13 & 13 & Pass & function use \\ \hline
    $let\ f = (x)\ \rightarrow \cos(x)$ & & & & \\ 
    $f(0)$ & 1.000 & 1.000 & Pass & function composition \\ \hline
    $let\ f = (x)\ \rightarrow \cos(x)$ & & & & \\ 
    $f(0) + f(0)$ & 2.000 & 2.000 & Pass & function composition \\ \hline
    $let f = () \rightarrow 5$ & & & & \\ 
    $f()$ & 5 & 5 & Pass & function no args \\ \hline
    $let f = () \rightarrow 5$ & & & & \\
    $let fa = (x) \rightarrow x * 4$ & & & & \\
    $f() + fa(3)$ & 17 & 17 & Pass & function no args \\ \hline
    $let f = () \rightarrow 5$ & & & & \\
    $let fa = (x) \rightarrow x * 4$ & & & & \\
    $fa(f())$ & 20 & 20 & Pass & function composition \\ \hline
    $let (|>) = (x, f) \rightarrow f(x)$ & & & & \\
    $5 |> (x) \rightarrow x * 4$ & 20 & 20 & Pass & operator overloading \\ \hline
    $let (|>) = (x, f) \rightarrow f(x)$ & & & & \\
    $5 |> (x) \rightarrow x * 4 |> (x) \rightarrow x + 3$ & 23 & 23 & Pass & operator overloading \\ \hline
    \end{tabular}
    \label{tab:functions}
\end{table}

Further tests are written in the language itself.
An example is shown below:

\begin{minted}{fsharp}
// function tests
let f = (x) -> x^2
assert f(3) == 9, "Function failed"
assert f(3) + f(2) == 13, "Function with expression failed"
assert f(3) + f(2) == 13, "Function with expression failed"
\end{minted}

\section{Type inference testing}\label{sec:type-inference-testing}

Type inference tests can be found in table~\ref{tab:type-inference}.

\begin{table}[h]
    \caption{Type Inference tests. ResE is expected result and ResA is actual result. \\}
    \begin{tabular}{|p{1.8in}|p{0.5in}|p{0.4in}|p{0.6in}|p{1.4in}|} \hline
    Expression & ResE & ResA& Pass/Fail & Action/comment \\ \hline \hline
    $let\ x = 3$ & int & int & Pass & int \\ \hline
    $let\ x = 3.0$ & float & float & Pass & float \\ \hline
    $let\ x = 3.0 + 2$ & float & float & Pass & addition \\ \hline
    $let\ x = 3.0 + 2.0$ & float & float & Pass & \\ \hline
    $let\ x = 3 + 2.0$ & float & float & Pass & \\ \hline
    $let\ x = 3 + 2$ & int & int & Pass & \\ \hline
    $let\ x = (() => 3)()$ & int & int & Pass & \\ \hline
    $let\ x = (() => 3.0)()$ & float & float & Pass & \\ \hline
    $let\ x = (() => [1,2])()$ & List[float] & List[float] & Pass & \\ \hline
    $let\ x = (() => {a = 3})()$ & Record[a: int] & Record[a: int] & Pass & \\ \hline
    $let\ x = (() => {a = 3, b = 4})()$ & Record[a: int, b: int] & Record[a: int, b: int] & Pass & \\ \hline
    $let\ x = 4 : float$ & float & float & Pass & Casting \\ \hline
    $let\ x = 4 : int$ & int & int & Pass & \\ \hline
    $let\ x = 4.0 : int$ & int & int & Pass & Casting \\ \hline
    $let\ x = 4.0 : float$ & float & float & Pass & \\ \hline
    \end{tabular}
    \label{tab:type-inference}
\end{table}

\section{Compound data type testing}\label{sec:compound-data-type-testing}

Tests can be found in table~\ref{tab:compound}.

\begin{table}[h]
    \caption{Compound DT tests. Note that floating pointing values are accurate to three decimal places for the fractional part. ResE is expected result and ResA is actual result. \\}
    \begin{tabular}{|p{1.8in}|p{0.5in}|p{0.4in}|p{0.6in}|p{1.4in}|} \hline
    Expression & ResE & ResA& Pass/Fail & Action/comment \\ \hline \hline
    $[1,2,3]$ & [1,2,3] & [1,2,3] & Pass & list \\ \hline
    $[1,2,3][1]$ & 2 & 2 & Pass & list index \\ \hline
    $[1,2,3][1..2]$ & [2] & [2] & Pass & list index range \\ \hline
    $[1,2,3][1..]$ & [2,3] & [2,3] & Pass & list index range \\ \hline
    $[1,2,3][..2]$ & [1] & [1] & Pass & list index range \\ \hline
    $[1..3]$ & [1,2,3] & [1,2,3] & Pass & list range \\ \hline
    $[1..3][1]$ & 2 & 2 & Pass & list range index \\ \hline
    $[1..3][1..2]$ & [2] & [2] & Pass & list range index range \\ \hline
    $\{ a = 3 \}$ & { a = 3 } & { a = 3 } & Pass & record \\ \hline
    $\{ a = 3 \}.a$ & 3 & 3 & Pass & record index \\ \hline
    $\{ a = 3 \}.b$ & Error & Error & Pass & record index \\ \hline
    $\{ a = 3, b = 4 \}.b$ & 4 & 4 & Pass & record index \\ \hline
    $\{ a = 3, b = 4 \}.a$ & 3 & 3 & Pass & record index \\ \hline
    $\{ a = 3, b = 4 \}.a + { a = 3, b = 4 }.b$ & 7 & 7 & Pass & record index \\ \hline
    \end{tabular}
    \label{tab:compound}
\end{table}

\section{Control flow testing}\label{sec:control-flow-testing}

Tests can be found in table~\ref{tab:control}.

\begin{table}[h]
    \caption{Control Flow tests. Note that floating pointing values are accurate to three decimal places for the fractional part. ResE is expected result and ResA is actual result. \\}
    \begin{tabular}{|p{1.8in}|p{0.5in}|p{0.4in}|p{0.6in}|p{1.4in}|} \hline
    Expression & ResE & ResA& Pass/Fail & Action/comment \\ \hline \hline
    $if\ 5 == 4\ then\ 3\ else\ 2$ & 2 & 2 & Pass & false \\ \hline
    $if\ 5 == 5\ then\ 3\ else\ 2$ & 3 & 3 & Pass & true \\ \hline
    $if\ 5\ then\ 3\ else\ 2$ & Error & Error & Pass & invalid type \\ \hline
    $if\ 5 == 5\ then\ 3$ & Error & Error & Pass & missing else \\ \hline
    $if\ true\ then\ print("here")$ & here & here & Pass & print no else (unit branch) \\ \hline
    $if\ false\ then\ 1\ else\ 2$ & 2 & 2 & Pass & false \\ \hline
    $let\ rec\ fact=(n)\ \rightarrow if\ n \leq 1\ then\ 1\ else\ n * fact(n-1)$ & & & & \\
    $fact(5)$ & 120 & 120 & Pass & factorial \\ \hline
    \end{tabular}
    \label{tab:control}
\end{table}

\section{GUI testing}\label{sec:gui-testing}

Tests can be found in table~\ref{tab:gui}.

\begin{table}[h]
    \caption{GUI Tests}
    \begin{tabular}{|p{1.8in}|p{0.5in}|p{0.4in}|p{0.6in}|p{1.4in}|} \hline
    Action & ResE & ResA& Pass/Fail & Action/comment \\ \hline \hline
        Click load code & Run code & Run code & Pass & Load code \\ \hline
        Click open file & Open file dialog & Open file dialog & Pass & Open file dialog \\ \hline
        Click open file & & & & \\ 
        Chose file & Opens file & Opens file & Pass & Open file \\ \hline
        Click open notebook & New notebook & New notebook & Pass & Open notebook \\ \hline
        Type in repl & & & & \\
        Click run & Run code & Run code & Pass & Output result \\ \hline
        Plot some data & Plot window & Plot window & Pass & Plot data \\ \hline
        Plot some data & & & & \\
        Run code & & & & \\
        Type in plot repl & Display plot & Display plot & Pass & Plot data \\ \hline
        Open notebook & & & & \\
        Add code & & & & \\
        Run code & Output result & Output result & Pass & Output result \\ \hline
        Open notebook & & & & \\
        Add code & & & & \\
        Run code & & & & \\
        Add code & & & & \\
        Run code with previous code use & Output result & Output result & Pass & Output result \\ \hline
        Open notebook & & & & \\
        Import file & Previous code & Previous code & Pass & Import code \\ \hline
        Open notebook & & & & \\
        Add code & & & & \\
        Save notebook & Save file & Save file & Pass & Save file \\ \hline
    \end{tabular}
    \label{tab:gui}
\end{table}

\section{Plot testing}\label{sec:plot-testing}

Tests can be found in table~\ref{tab:plot-tests}.

\begin{table}[h]
    \caption{Plot Tests}
\begin{tabular}{|p{1.8in}|p{0.5in}|p{0.4in}|p{0.6in}|p{1.4in}|} \hline
Plot & ResE & ResA& Pass/Fail & Action/comment \\ \hline \hline
$let\ x = [1..10] : [float]$ & & & & \\
$let\ y = map(x, (x) -> x^2)$ & & & & \\
$let\ data = \{ title = "Example Plot", x = x, y = y, ptype = "scatter" \}$ & & & & \\
$plot(data)$ & $f(x) = x^2$ & $f(x) = x^2$ & Pass & scatter plot \\ \hline
$let\ f = (x) -> x^2$ & & & & \\
$plotFunc("Example Plot", f)$ & $f(x) = x^2$ & $f(x) = x^2$ & Pass & plot function \\ \hline
$let\ f = (x) -> \cos(x)$ & & & & \\
$plotFunc("Example Plot", f)$ & $f(x) = \cos(x)$ & $f(x) = \cos(x)$ & Pass & plot function \\ \hline
$plotFunc("Example Plot", \cos)$ & $f(x) = \cos(x)$ & $f(x) = \cos(x)$ & Pass & plot function \\ \hline
$let\ f = (x) -> \tan(x)$ & & & & \\
$plotFunc($ & & & & \\
$"Example Plot", f)$ & $f(x) = \tan(x)$ & $f(x) = \tan(x)$ & Pass, but not infinite asymptotes & No joined asymptotes \\ \hline
$let\ f = (x) -> 1 / x$ & & & & \\
$plotFunc($ & & & & \\
$"Example Plot", f)$ & $f(x) = 1/x$ & $f(x) = 1/x$ & Pass, but not infinite asymptotes & No joined asymptotes \\ \hline
\end{tabular}
\label{tab:plot-tests}
\end{table}


\chapter{Syntax}\label{ch:other-stuff}

Is needed?
    
\chapter{Algorithms}\label{ch:algorithms}

This chapter will showcase some of the maths focused algorithms used in the project.

\section{Root finding}\label{sec:root-finding}

Both the bisection and Newton-Raphson methods are implemented.

\begin{algorithm}
    \caption{Bisection method}
    \begin{algorithmic}
        \Function{bisection}{f, a, b, tol}
            \State $c \gets (a + b) / 2$
            \While{$|f(c)| > tol$}
                \If{$f(a) \cdot f(c) < 0$}
                    \State $b \gets c$
                \Else
                    \State $a \gets c$
                \EndIf
                \State $c \gets (a + b) / 2$
            \EndWhile
            \State \Return $c$
        \EndFunction
    \end{algorithmic}\label{alg:algorithm2}
\end{algorithm}

\begin{algorithm}
    \caption{Newton-Raphson method}
    \begin{algorithmic}
        \Function{newtonRaphson}{f, df, x0, tol}
            \State $x1 \gets x0 - f(x0) / df(x0)$
            \While{$|f(x1)| > tol$}
                \State $x0 \gets x1$
                \State $x1 \gets x0 - f(x0) / df(x0)$
            \EndWhile
            \State \Return $x1$
        \EndFunction
    \end{algorithmic}\label{alg:algorithm3}
\end{algorithm}

\section{Calculus}\label{sec:integration}
In the language, pure functions are represented in a symbolic expression DSL, making it easy to differentiate and 
integrate functions, as well as find tangent lines or the tailor series of a function.
The symbolic expression DSL created was inspired by the systems found at \citet{symbolicExprHask}, 
\citet{symbolicExprPython} and \citet{symbolicExprF}.

\begin{algorithm}
    \caption{Differentiation}
    \begin{algorithmic}
        \Function{differentiate}{f}
        \If{$f$ is a constant}
            \State \Return $0$
        \ElsIf{$f$ is a variable}
            \State \Return $1$
        \ElsIf{$f$ is a sum}
            \State \Return $\text{differentiate}(f_1) + \text{differentiate}(f_2)$
        \ElsIf{$f$ is a product}
            \State \Return $f_1 \cdot \text{differentiate}(f_2) + f_2 \cdot \text{differentiate}(f_1)$
        \ElsIf{$f$ is a power}
            \State \Return $n \cdot x^{n-1}$
        \ElsIf{$f$ is cos}
            \State \Return $-\sin(x)$
        \ElsIf{$f$ is sin}
            \State \Return $\cos(x)$
        \ElsIf{$f$ is tan}
            \State \Return $\sec^2(x)$
        \EndIf

        \EndFunction
    \end{algorithmic}\label{alg:algorithm4}
\end{algorithm}

\section{Vector operations}\label{sec:vector-ops}

Functions used to find the cross and dot products of two vectors are implemented, as well as standard arithmetic
operations.

\begin{algorithm}
    \caption{Cross product}
    \begin{algorithmic}
        \Function{crossProduct}{v1, v2}
            \State $x \gets v1[1] \cdot v2[3] - v1[3] \cdot v2[2]$
            \State $y \gets v1[3] \cdot v2[1] - v1[1] \cdot v2[3]$
            \State $z \gets v1[1] \cdot v2[2] - v1[2] \cdot v2[1]$
            \State \Return $[x, y, z]$
        \EndFunction
    \end{algorithmic}\label{alg:algorithm6}
\end{algorithm}

\begin{algorithm}
    \caption{Dot product}
    \begin{algorithmic}
        \Function{dotProduct}{v1, v2}
            \State $sum \gets 0$
            \For{$i \gets 1$ to $n$}
                \State $sum \gets sum + v1[i] \cdot v2[i]$
            \EndFor
            \State \Return $sum$
        \EndFunction
    \end{algorithmic}\label{alg:algorithm7}
\end{algorithm}

\section{Matrix operations}\label{sec:matrix-ops}

The language has built-in functions for finding the transpose, determinant and inverse of a matrix.

\begin{algorithm}
    \caption{Transpose}
    \begin{algorithmic}
        \Function{transpose}{m}
            \State $rows \gets \text{length}(m)$
            \State $cols \gets \text{length}(m[1])$
            \State $trans \gets \text{emptyMatrix}(cols, rows)$
            \For{$i \gets 1$ to $rows$}
                \For{$j \gets 1$ to $cols$}
                    \State $trans[j][i] \gets m[i][j]$
                \EndFor
            \EndFor
            \State \Return $trans$
        \EndFunction
    \end{algorithmic}\label{alg:algorithm8}
\end{algorithm}

\begin{algorithm}
    \caption{Determinant}
    \begin{algorithmic}
        \Function{determinant}{m}
            \State $rows \gets \text{length}(m)$
            \State $cols \gets \text{length}(m[1])$
            \If{$rows \neq cols$}
                \State \Return $0$
            \EndIf
            \If{$rows = 2$}
                \State \Return $m[1][1] \cdot m[2][2] - m[1][2] \cdot m[2][1]$
            \EndIf
            \State $\det \gets 0$
            \For{$i \gets 1$ to $rows$}
                \State $\det \gets \det + m[1][i] \cdot \text{cofactor}(m, 1, i)$
            \EndFor
            \State \Return $\det$
        \EndFunction
    \end{algorithmic}\label{alg:algorithm9}
\end{algorithm}

\begin{algorithm}
    \caption{Inverse}
    \begin{algorithmic}
        \Function{inverse}{m}
            \State $rows \gets \text{length}(m)$
            \State $cols \gets \text{length}(m[1])$
            \State $\det \gets \text{determinant}(m)$
            \If{$\det = 0$}
                \State \Return \text{error}
            \EndIf
            \State $inv \gets \text{emptyMatrix}(rows, cols)$
            \For{$i \gets 1$ to $rows$}
                \For{$j \gets 1$ to $cols$}
                    \State $inv[i][j] \gets \text{cofactor}(m, i, j) / \det$
                \EndFor
            \EndFor
            \State \Return $inv$
        \EndFunction
    \end{algorithmic}\label{alg:algorithm10}
\end{algorithm}

\end{document}

