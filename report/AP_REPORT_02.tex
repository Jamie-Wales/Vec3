
\documentclass[a4paper, oneside, 11pt]{report}
\usepackage{epsfig,pifont,float,multirow,amsmath,amssymb}
\newcommand{\mc}{\multicolumn{1}{c|}}
\newcommand{\mb}{\mathbf}
\newcommand{\mi}{\mathit}
\newcommand{\oa}{\overrightarrow}
\newcommand{\bs}{\boldsymbol}
\newcommand{\ra}{\rightarrow}
\newcommand{\la}{\leftarrow}
\usepackage{algorithm}
\usepackage{algorithmic}
\usepackage{natbib}
\usepackage{algorithmicx}
\topmargin = 0pt
\voffset = -80pt
\oddsidemargin = 15pt
\textwidth = 425pt
\textheight = 750pt

\begin{document}

\begin{titlepage}
\begin{center}
\rule{12cm}{1mm} \\
\vspace{1cm}
{\large  CMP-6048A/7009A Advanced Programming} %Delete as appropriate
\vspace{7.5cm}
\\{\Large Project Report - 13 January 2025}
\vspace{1.5cm}
\\{\LARGE Vec3 Maths Bytecode Interpreter} % You can add to this title of modify it if you wish
\vspace{1.0cm}
\\{\Large Group members: \\ Jamie Wales, Jacob Edwards.\ }
\vspace{10.0cm}
\\{\large School of Computing Sciences, University of East Anglia}
\\ \rule{12cm}{0.5mm}
\\ \hspace{8.5cm} {\large Version 2.0}
\end{center}
\end{titlepage}


\setcounter{page}{1}

\begin{abstract}
Vec3 is a bytecode interpreted maths language complete with a GUI, plotting, full static type inference and various 
maths functions.

The language is designed to be simple to use and understand, with a focus on strict mathematical expressions and 
plotting, but with more powerful constructs such as recursive bindings, first class functions and static type inference.

The language and GUI are written in F\#.
\end{abstract}

\chapter{Introduction}\label{ch:intro}

The introduction should be brief and comprise the following:

\section{Project statement}\label{sec:project-statement}


\section{Aims and objectives}\label{sec:aims-and-objectives}

\chapter{Background}\label{ch:background}

\chapter{Development History}\label{ch:devhist}

\section{Sprint 1: Basic expressions}\label{sec:sprint-1:-basic-expressions}

This sprint focused on implementing a lexer and parser for the language, with precedence rules for the arithmetic 
operators, parsed with Pratt parsing.

\subsection{Grammar in BNF}\label{subsec:grammar-in-bnf1}

\begin{verbatim}
<expr> ::= <term> | <term> "+" <expr> | <term> "-" <expr>
<term> ::= <factor> | <factor> "*" <term> | <factor> "/" <term>
<factor> ::= <number> | "(" <expr> ")"
<number> ::= <int> | <float>
<int> ::= <digit> | <digit> <int>
<float> ::= <int> "." <int>
<digit> ::= "0" | "1" | "2" | "3" | "4" | "5" | "6" | "7" | "8" | "9"
\end{verbatim}

\section{Sprint 2: Variable assignment}\label{sec:variable-assignment}

\subsection{Grammar in BNF}\label{subsec:grammar-in-bnf2}

\begin{verbatim}
<stmtlist> ::= <stmt> | <stmt> <stmtlist>
<stmt> ::= <expr> | "let" <identifier> "=" <expr>

<expr> ::= <term> | <term> "+" <expr> | <term> "-" <expr>
<term> ::= <factor> | <factor> "*" <term> | <factor> "/" <term> | <factor> "%" <term>
<factor> ::= <number> | <identifier> | "(" <expr> ")" | <factor> "^" <factor>

<number> ::= <int> | <float>

<int> ::= <digit> | <digit> <int>
<float> ::= <int> "." <int>
<digit> ::= "0" | "1" | "2" | "3" | "4" | "5" | "6" | "7" | "8" | "9"

<identifier> ::= <letter> | <letter> <identifier>
<letter> ::= "a" | "b" | "c" | "d" | "e" | "f" | "g" | "h" | "i" | "j" | "k" | "l" | "m" | "n" | "o" | "p" | "q" | "r" | "s" | "t" | "u" | "v" | "w" | "x" | "y" | "z"
\end{verbatim}

\section{Sprint 3:; Interpreter}\label{sec:interpreter}

\section{Sprint 4: Functions}\label{sec:functions}

\subsection{Grammar in BNF}\label{subsec:grammar-in-bnf3}

\begin{verbatim}
<stmtlist> ::= <stmt> | <stmt> <stmtlist>
<stmt> ::= <expr> | "let" <identifier> "=" <expr>
<expr> ::= <term> | <term> "+" <expr> | <term> "-" <expr>
<term> ::= <factor> | <factor> "*" <term> | <factor> "/" <term> | <factor> "%" <term>
<factor> ::= <number> | <identifier> | "(" <expr> ")" | <factor> "^" <factor> | <identifier> "(" <exprlist> ")" | <lambda>
<lambda> ::= "(" <exprlist> ")" "->" <expr>

<number> ::= <int> | <float>
<int> ::= <digit> | <digit> <int>
<float> ::= <int> "." <int>

<digit> ::= "0" | "1" | "2" | "3" | "4" | "5" | "6" | "7" | "8" | "9"
<identifier> ::= <letter> | <letter> <identifier>
<letter> ::= "a" | "b" | "c" | "d" | "e" | "f" | "g" | "h" | "i" | "j" | "k" | "l" | "m" | "n" | "o" | "p" | "q" | "r" | "s" | "t" | "u" | "v" | "w" | "x" | "y" | "z"
<exprlist> ::= <expr> | <expr> "," <exprlist>
	
\end{verbatim}

\section{Sprint 5: Static type checking}\label{sec:static-type-checking}

\section{Sprint 6: Bytecode}\label{sec:bytecode}

\section{Sprint 7: GUI}\label{sec:gui}

\section{Sprint 8: Plotting}\label{sec:plotting1}

\section{Sprint 9: Optimisation}\label{sec:optimisation1}

\chapter{Final deliverable}\label{ch:impl}

\section{Final BNF}\label{sec:final-bnf}

\section{Final GUI}\label{sec:final-gui}

\section{Lexer}\label{sec:lexer}

\section{Parser}\label{sec:parser}

\section{Expression}\label{sec:expression}

\section{Type Inference}\label{sec:type-inference}

\section{Optimisation}\label{sec:optimisation}

\section{Compiler}\label{sec:compiler}

\section{Virtual Machine}\label{sec:virtual-machine}

\section{Plotting}\label{sec:plotting}

\section{Code architecture}\label{sec:code-architecture}

\chapter{Discussion, conclusion and future work}\label{ch:discussion-conclusion-and-future-work}

\bibliographystyle{apalike}
\bibliography{References}


\appendix
\chapter{Contributions}\label{ch:contributions}

50/50

\chapter{Testing}\label{ch:test}

\section{Lexer testing}\label{sec:lexer-testing}

\section{Parser testing}\label{sec:parser-testing}

\section{Expression testing}\label{sec:arithmetic-expression-testing}

\section{Variable assignment testing}\label{sec:variable-assignment-testing}

\section{Function testing}\label{sec:function-testing}

\section{GUI testing}\label{sec:gui-testing}

\section{Plot testing}\label{sec:plot-testing}

\chapter{Other stuff}\label{ch:other-stuff}
	
	
\end{document}

